% Activate the following line by filling in the right side. If for example the name of the root file is Main.tex, write
% "...root = Main.tex" if the chapter file is in the same directory, and "...root = ../Main.tex" if the chapter is in a subdirectory.

%!TEX root =  ../../Tesis.tex
\chapter{C�lculo de las tasas de sustituciones de mtDNA}\label{apendice:pairwise}

% Tabla Data 1
\LTcapwidth=\textwidth
\begin{longtable}{l l l l}
	\caption{\label{TablaPairwise} \small{Pares de especies cercanas empleadas para el an�lisis de tasas de sustituciones en el genoma mitocondrial. En el cap�tulo \ref{cap:surfbur} (p�gina \pageref{calculo:sustituciones}), se describe el proceso llevado a cabo para determinar los valores de $d_N$ y $d_S$, a partir de las secuencias pertenecientes a las especies mostradas en esta tabla.}}\\
	\hline
	no. & Especie 1 & Especie 2 & Orden\\
	\hline
	\endfirsthead
	\hline
	no. & Especie 1 & Especie 2 & Orden\\
	\hline
	\endhead
		\hline
		\endlastfoot
1 & \textit{Chrysochloris asiatica} & \textit{Eremitalpa granti} & Afrosoricida\\ 
2 & \textit{Ammotragus lervia} & \textit{Capra hircus} & Artiodactyla\\ 
3 & \textit{Bos indicus} & \textit{Bos taurus} & Artiodactyla\\ 
4 & \textit{Camelus bactrianus} & \textit{Camelus dromedarius} & Artiodactyla\\ 
5 & \textit{Cervus elaphus} & \textit{Cervus unicolor} & Artiodactyla\\ 
6 & \textit{Muntiacus muntjak} & \textit{Muntiacus reevesi} & Artiodactyla\\ 
7 & \textit{Phacochoerus africanus} & \textit{Sus scrofa} & Artiodactyla\\ 
8 & \textit{Arctocephalus forsteri} & \textit{Phocarctos hookeri} & Carnivora\\ 
9 & \textit{Canis latrans} & Canis \textit{lupus familiaris} & Carnivora\\ 
10 & \textit{Enhydra lutris} & \textit{Lutra lutra} & Carnivora\\ 
11 & \textit{Eumetopias jubatus} & \textit{Zalophus californianus} & Carnivora\\ 
12 & \textit{Halichoerus grypus} & \textit{Phoca sibirica} & Carnivora\\ 
13 & \textit{Leptonychotes weddellii} & \textit{Lobodon carcinophaga} & Carnivora\\ 
14 & \textit{Martes melampus} & \textit{Martes zibellina} & Carnivora\\ 
15 & \textit{Panthera pardus} & \textit{Panthera tigris} & Carnivora\\ 
16 & \textit{Phoca fasciata} & \textit{Phoca groenlandica} & Carnivora\\ 
17 & \textit{Phoca largha} & \textit{Phoca vitulina} & Carnivora\\ 
18 & \textit{Ursus arctos} & \textit{Ursus maritimus} & Carnivora\\ 
19 & \textit{Balaenoptera acutorostrata} & \textit{Balaenoptera bonaerensis} & Cetacea\\ 
20 & \textit{Balaenoptera brydei} & \textit{Balaenoptera edeni} & Cetacea\\ 
21 & \textit{Balaenoptera physalus} & \textit{Megaptera novaeangliae} & Cetacea\\ 
22 & \textit{Berardius bairdii} & \textit{Hyperoodon ampullatus} & Cetacea\\ 
23 & \textit{Eubalaena australis} & \textit{Eubalaena japonica} & Cetacea\\ 
24 & \textit{Kogia breviceps} & \textit{Physeter catodon} & Cetacea\\ 
25 & \textit{Monodon monoceros} & \textit{Phocoena phocoena} & Cetacea\\ 
26 & \textit{Artibeus jamaicensis} & \textit{Mystacina tuberculata} & Chiroptera\\ 
27 & \textit{Chalinolobus tuberculatus} & \textit{Pipistrellus abramus} & Chiroptera\\ 
28 & \textit{Pteropus dasymallus} & \textit{Pteropus scapulatus} & Chiroptera\\ 
29 & \textit{Rhinolophus monoceros} & \textit{Rhinolophus pumilus} & Chiroptera\\ 
30 & \textit{Dasyurus hallucatus} & \textit{Phascogale tapoatafa} & Dasyuromorphia\\ 
31 & \textit{Myrmecobius fasciatus} & \textit{Sminthopsis douglasi} & Dasyuromorphia\\ 
32 & \textit{Metachirus nudicaudatus} & \textit{Thylamys elegans} & Didelphimorphia\\ 
33 & \textit{Dactylopsila trivirgata} & \textit{Petaurus breviceps} & Diprodontiae\\ 
34 & \textit{Phalanger interpositus} & \textit{Trichosurus vulpecula} & Diprodontiae\\ 
35 & \textit{Phascolarctos cinereus} & \textit{Vombatus ursinus} & Diprodontiae\\ 
36 & \textit{Pseudocheirus peregrinus} & \textit{Tarsipes rostratus} & Diprodontiae\\ 
37 & \textit{Erinaceus europaeus} & \textit{Hemiechinus auritus} & Erinaceomorpha\\ 
38 & \textit{Dendrohyrax dorsalis} & \textit{Procavia capensis} & Hyracoidea\\ 
39 & \textit{Ochotona collaris} & \textit{Ochotona princeps} & Lagomorpha\\ 
40 & \textit{Ornithorhynchus anatinus} & \textit{Tachyglossus aculeatus} & Monotremata\\ 
41 & \textit{Isoodon macrourus} & \textit{Macrotis lagotis} & Paramelemorphia\\ 
42 & \textit{Ceratotherium simum} & \textit{Rhinoceros unicornis} & Perissodactyla\\ 
43 & \textit{Bradypus tridactylus} & \textit{Choloepus didactylus} & Pilosa\\ 
44 & \textit{Colobus guereza} & \textit{Procolobus badius} & Primates\\ 
45 & \textit{Eulemur fulvus fulvus} & \textit{Eulemur fulvus mayottensis} & Primates\\ 
46 & \textit{Macaca sylvanus} & \textit{Papio hamadryas} & Primates\\ 
47 & \textit{Nasalis larvatus} & \textit{Pygathrix roxellana} & Primates\\ 
48 & \textit{Pongo abelii} & \textit{Pongo pygmaeus} & Primates\\ 
49 & \textit{Tarsius bancanus} & \textit{Tarsius syrichta} & Primates\\ 
50 & \textit{Elephas maximus} & \textit{Mammuthus primigenius} & Proboscidea\\ 
51 & \textit{Microtus kikuchii} & \textit{Microtus rossiaemeridionalis} & Rodentia\\ 
52 & \textit{Rattus norvegicus} & \textit{Rattus rattus} & Rodentia\\ 
53 & \textit{Sciurus vulgaris} & \textit{Thryonomys swinderianus} & Rodentia\\ 
54 & \textit{Galemys pyrenaicus} & \textit{Urotrichus talpoides} & Soricomorpha\\
	\hline
\end{longtable}

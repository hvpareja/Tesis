% Activate the following line by filling in the right side. If for example the name of the root file is Main.tex, write
% "...root = Main.tex" if the chapter file is in the same directory, and "...root = ../Main.tex" if the chapter is in a subdirectory.

%!TEX root =  ../../Tesis.tex
\chapter{Datos}\label{apendice:datos}
\section{Datos del cap�tulo 1}
\footnotetext[1]{El <<\textit{accession number}>> es el c�digo que identifica el genoma en la base de datos de NCBI (\url{http://www.ncbi.nlm.nih.gov/}).}
% Tabla Data 1
\LTcapwidth=\textwidth
\begin{longtable}{l l c c c c}
	\caption{\label{TablaDatos1} \small{Genomas empleados en el trabajo recogido en el cap�tulo \ref{cap:Met}. En la tecera columna se indica el \textit{accession number}\protect\footnotemark[1] de la especie. La muestra se compone de los genomas mitocondriales de 173 especies de mam�feros, las cuales fueron clasficadas como longevas (L) o no longevas (S) dependiendo de si el $log{MLSP}$ es mayor o menor a 2.46, valor que se corresponde con 228.4 meses. Adicionalmente, se muestra el logaritmo de la biomasa ($log{BM}$) en la 6� columna.}}\\
	\hline
	no. & Especie & \emph{Accession number} & Tipo & $log{MLSP}$ & $log{BM}$ \\
	\hline
	\endfirsthead
	\hline
	no. & Especie & \emph{Accession number} & Tipo & $log{MLSP}$ & $log{BM}$ \\
	\hline
	\endhead
		\hline
		\endlastfoot
1  & \textit{Acinonyx jubatus} & NC\_005212 & S & 2.3909 & 4.6733 \\
2  & \textit{Ailuropoda melanoleuca} & NC\_009492 & S & 2.6450 & 5.0107 \\
3  & \textit{Ailurus fulgens} & NC\_011124 & S & 2.3579 & 3.6990 \\
4  & \textit{Ammotragus lervia} & NC\_009510 & S & 2.4156 & 5.0212 \\
5  & \textit{Arctocephalus pusillus} & NC\_008417 & S & 2.5857 & 5.3010 \\
6  & \textit{Artibeus jamaicensis} & NC\_002009 & L & 2.3625 & 1.6484 \\
7  & \textit{Balaena mysticetus} & NC\_005268 & S & 3.4035 & 7.9624 \\
8  & \textit{Balaenoptera acutorostrata} & NC\_005271 & S & 2.7782 & 6.9191 \\
9  & \textit{Balaenoptera borealis} & NC\_006929 & S & 2.9484 & 7.3010 \\
10  & \textit{Balaenoptera edeni} & NC\_007938 & S & 2.9365 & 7.3010 \\
11  & \textit{Balaenoptera musculus} & NC\_001601 & S & 3.1206 & 8.2788 \\
12  & \textit{Balaenoptera physalus} & NC\_001321 & S & 3.1361 & 7.8451 \\
13  & \textit{Berardius bairdii} & NC\_005274 & S & 3.0035 & 7.0000 \\
14  & \textit{Bos grunniens} & NC\_006380 & S & 2.4991 & 5.7782 \\
15  & \textit{Bos taurus} & NC\_006853 & S & 2.3802 & 5.7875 \\
16  & \textit{Bubalus bubalis} & NC\_006295 & S & 2.6220 & 5.9633 \\
17  & \textit{Camelus bactrianus} & NC\_009628 & S & 2.6282 & 5.7653 \\
18  & \textit{Camelus dromedarius} & NC\_009849 & S & 2.5325 & 5.7526 \\
19  & \textit{Canis latrans} & NC\_008093 & S & 2.4176 & 4.0959 \\
20  & \textit{Canis lupus} & NC\_008092 & S & 2.3930 & 4.6721 \\
21  & \textit{Capra hircus} & NC\_005044 & S & 2.3972 & 4.7076 \\
22  & \textit{Cavia porcellus} & NC\_000884 & S & 2.1584 & 2.8325 \\
23  & \textit{Cebus albifrons} & NC\_002763 & S & 2.6856 & 3.3424 \\
24  & \textit{Ceratotherium simum} & NC\_001808 & S & 2.7324 & 6.4370 \\
25  & \textit{Cervus elaphus} & NC\_007704 & S & 2.5775 & 5.2122 \\
26 & \textit{Cervus nippon centralis} & NC\_006993 & S & 2.4991 & 4.7243 \\
27 & \textit{Cervus unicolor swinhoei} & NC\_008414 & S & 2.5008 & 5.2553 \\
28  & \textit{Chlorocebus aethiops} & NC\_007009 & S & 2.5677 & 3.6128 \\
29  & \textit{Choloepus didactylus} & NC\_006924 & S & 2.6450 & 3.7574 \\
30  & \textit{Colobus guereza} & NC\_006901 & S & 2.6232 & 4.0086 \\
31  & \textit{Cricetulus griseus} & NC\_007936 & L & 1.8035 & 1.7782 \\
32  & \textit{Cystophora cristata} & NC\_008427 & S & 2.6232 & 5.5051 \\
33  & \textit{Dactylopsila trivirgata} & NC\_008134 & S & 2.0615 & 2.6160 \\
34  & \textit{Dasypus novemcinctus} & NC\_001821 & S & 2.4275 & 3.6721 \\
35  & \textit{Dasyurus hallucatus} & NC\_007630 & S & 1.8500 & 2.6848 \\
36  & \textit{Daubentonia madagascariensis} & NC\_010299 & S & 2.4465 & 3.3979 \\
37  & \textit{Didelphis virginiana} & NC\_001610 & S & 1.8987 & 3.3959 \\
38  & \textit{Distoechurus pennatus} & NC\_008145 & L & 1.3802 & 1.7251 \\
39  & \textit{Dugong dugon} & NC\_003314 & S & 2.9425 & 5.5315 \\
40  & \textit{Echinops telfairi} & NC\_002631 & L & 2.3579 & 2.1038 \\
41 & \textit{Echymipera rufescens australis} & NC\_007632 & S & 1.5705 & 2.7896 \\
42  & \textit{Elaphodus cephalophus} & NC\_008749 & S & 2.4352 & 4.5250 \\
43  & \textit{Elephantulus sp} & NC\_004921 & L & 1.9868 & 1.7033 \\
44  & \textit{Elephas maximus} & NC\_005129 & S & 2.8954 & 6.5250 \\
45  & \textit{Enhydra lutris} & NC\_009692 & S & 2.5105 & 4.4232 \\
46  & \textit{Equus asinus} & NC\_001788 & S & 2.7513 & 5.2553 \\
47  & \textit{Equus caballus} & NC\_001640 & S & 2.8351 & 5.6021 \\
48  & \textit{Erinaceus europaeus} & NC\_002080 & S & 2.1474 & 2.8814 \\
49  & \textit{Eschrichtius robustus} & NC\_005270 & S & 2.9657 & 7.4548 \\
50  & \textit{Eubalaena australis} & NC\_006930 & S & 2.9243 & 7.6532 \\
51  & \textit{Eubalaena japonica} & NC\_006931 & S & 2.9243 & 7.8451 \\
52 & \textit{Eulemur fulvus fulvus} & NC\_012766 & S & 2.6294 & 3.4983 \\
53 & \textit{Eulemur macaco macaco} & NC\_012771 & S & 2.6379 & 3.3979 \\
54  & \textit{Eulemur mongoz} & NC\_010300 & S & 2.6379 & 3.3010 \\
55  & \textit{Eumetopias jubatus} & NC\_004030 & S & 2.5951 & 5.4914 \\
56  & \textit{Felis catus} & NC\_001700 & S & 2.5563 & 3.6675 \\
57  & \textit{Galemys pyrenaicus} & NC\_008156 & L & 1.7782 & 1.7709 \\
58  & \textit{Glis glis} & NC\_001892 & L & 2.0170 & 2.1903 \\
59  & \textit{Gorilla gorilla} & NC\_001645 & S & 2.8227 & 5.1847 \\
60  & \textit{Gulo gulo} & NC\_009685 & S & 2.3692 & 4.2304 \\
61  & \textit{Halichoerus grypus} & NC\_001602 & S & 2.7116 & 5.2455 \\
62  & \textit{Helarctos malayanus} & NC\_009968 & S & 2.6343 & 4.6628 \\
63  & \textit{Hemiechinus auritus} & NC\_005033 & S & 1.9600 & 2.5575 \\
64  & \textit{Herpestes javanicus} & NC\_006835 & S & 2.2993 & 2.7853 \\
65  & \textit{Hippopotamus amphibius} & NC\_000889 & S & 2.8659 & 6.4728 \\
66  & \textit{Homo sapiens} & NC\_012920 & S & 3.0792 & 4.7926 \\
67  & \textit{Hydrurga leptonyx} & NC\_008425 & S & 2.4942 & 5.5798 \\
68  & \textit{Hylobates lar} & NC\_002082 & S & 2.8274 & 3.6990 \\
69  & \textit{Hyperoodon ampullatus} & NC\_005273 & S & 2.6474 & 6.6335 \\
70  & \textit{Inia geoffrensis} & NC\_005276 & S & 2.5747 & 5.1335 \\
71  & \textit{Isoodon macrourus} & NC\_002746 & S & 1.9117 & 3.1906 \\
72  & \textit{Jaculus jaculus} & NC\_005314 & L & 1.9425 & 1.8129 \\
73  & \textit{Kogia breviceps} & NC\_005272 & S & 2.3096 & 5.6180 \\
74  & \textit{Lama pacos} & NC\_002504 & S & 2.4908 & 4.8129 \\
75  & \textit{Lemur catta} & NC\_004025 & S & 2.6509 & 3.4624 \\
76  & \textit{Leptonychotes weddellii} & NC\_008424 & S & 2.4771 & 5.6684 \\
77  & \textit{Lepus europaeus} & NC\_004028 & S & 2.1086 & 3.5911 \\
78  & \textit{Lipotes vexillifer} & NC\_007629 & S & 2.4594 & 4.9754 \\
79  & \textit{Lobodon carcinophaga} & NC\_008423 & S & 2.6702 & 5.3820 \\
80  & \textit{Loxodonta africana} & NC\_000934 & S & 2.8921 & 6.6021 \\
81  & \textit{Lutra lutra} & NC\_011358 & S & 2.3393 & 4.0000 \\
82  & \textit{Macaca mulatta} & NC\_005943 & S & 2.6812 & 3.8096 \\
83  & \textit{Macaca sylvanus} & NC\_002764 & S & 2.5431 & 4.0000 \\
84  & \textit{Macropus robustus} & NC\_001794 & S & 2.4216 & 4.4669 \\
85  & \textit{Macroscelides proboscideus} & NC\_004026 & L & 2.0187 & 1.6021 \\
86  & \textit{Macrotis lagotis} & NC\_006520 & S & 2.0615 & 3.1119 \\
87  & \textit{Martes flavigula} & NC\_012141 & S & 2.2833 & 3.6021 \\
88  & \textit{Martes zibellina} & NC\_011579 & S & 2.3440 & 3.0969 \\
89  & \textit{Megaptera novaeangliae} & NC\_006927 & S & 3.0569 & 7.4771 \\
90  & \textit{Meles meles} & NC\_011125 & S & 2.3487 & 4.0969 \\
91  & \textit{Melursus ursinus} & NC\_009970 & S & 2.6016 & 4.9759 \\
92  & \textit{Metachirus nudicaudatus} & NC\_006516 & S & 1.5563 & 2.5263 \\
93  & \textit{Mirounga leonina} & NC\_008422 & S & 2.4409 & 6.2041 \\
94  & \textit{Mogera wogura} & NC\_005035 & L & 1.5843 & 1.8319 \\
95  & \textit{Monachus schauinslandi} & NC\_008421 & S & 2.5563 & 5.3483 \\
96  & \textit{Monodelphis domestica} & NC\_006299 & L & 1.7868 & 2.0170 \\
97  & \textit{Monodon monoceros} & NC\_005279 & S & 2.7782 & 6.0314 \\
98  & \textit{Muntiacus muntjak} & NC\_004563 & S & 2.3533 & 4.2788 \\
99  & \textit{Muntiacus reevesi} & NC\_004069 & S & 2.4447 & 4.1761 \\
100 & \textit{Mus musculus domesticus} & NC\_006914 & L & 1.6812 & 1.2430 \\
101  & \textit{Myrmecobius fasciatus} & NC\_011949 & S & 2.1206 & 2.6021 \\
102  & \textit{Mystacina tuberculata} & NC\_006925 & L & 1.9600 & 1.2553 \\
103  & \textit{Nannospalax ehrenbergi} & NC\_005315 & L & 2.2553 & 2.1553 \\
104  & \textit{Nasalis larvatus} & NC\_008216 & S & 2.4789 & 4.1004 \\
105  & \textit{Neofelis nebulosa} & NC\_008450 & S & 2.3758 & 4.3010 \\
106  & \textit{Neophoca cinerea} & NC\_008419 & S & 2.4612 & 5.2928 \\
107  & \textit{Nycticebus coucang} & NC\_002765 & S & 2.4908 & 3.0700 \\
108  & \textit{Ochotona collaris} & NC\_003033 & L & 1.8573 & 2.1139 \\
109  & \textit{Ochotona princeps} & NC\_005358 & L & 1.9243 & 2.0374 \\
110  & \textit{Ornithorhynchus anatinus} & NC\_000891 & S & 2.4333 & 2.8407 \\
111  & \textit{Orycteropus afer} & NC\_002078 & S & 2.5534 & 4.7005 \\
112  & \textit{Oryctolagus cuniculus} & NC\_001913 & S & 2.0334 & 3.3010 \\
113  & \textit{Ovis aries} & NC\_001941 & S & 2.4371 & 4.5340 \\
114  & \textit{Pan paniscus} & NC\_001644 & S & 2.8195 & 4.5623 \\
115  & \textit{Pan troglodytes} & NC\_001643 & S & 2.8530 & 4.6675 \\
116  & \textit{Panthera pardus} & NC\_010641 & S & 2.5153 & 4.7306 \\
117  & \textit{Panthera tigris} & NC\_010642 & S & 2.4991 & 5.1775 \\
118  & \textit{Papio hamadryas} & NC\_001992 & S & 2.6532 & 4.2385 \\
119  & \textit{Pecari tajacu} & NC\_012103 & S & 2.5775 & 4.3118 \\
120  & \textit{Perameles gunnii} & NC\_006521 & S & 1.8645 & 2.9227 \\
121  & \textit{Petaurus breviceps} & NC\_008135 & L & 2.3296 & 2.1038 \\
122  & \textit{Phacochoerus africanus} & NC\_008830 & S & 2.3993 & 4.9165 \\
123  & \textit{Phascogale tapoatafa} & NC\_006523 & L & 1.8500 & 2.1959 \\
124  & \textit{Phascolarctos cinereus} & NC\_008133 & S & 2.4236 & 3.6781 \\
125  & \textit{Phoca hispida} & NC\_008433 & S & 2.7419 & 4.9542 \\
126  & \textit{Phoca largha} & NC\_008430 & S & 2.7419 & 4.9590 \\
127  & \textit{Phoca sibirica} & NC\_008432 & S & 2.8274 & 5.0414 \\
128  & \textit{Phoca vitulina} & NC\_001325 & S & 2.7568 & 5.0212 \\
129  & \textit{Phocoena phocoena} & NC\_005280 & S & 2.3888 & 4.7202 \\
130  & \textit{Physeter catodon} & NC\_002503 & S & 2.9657 & 7.3617 \\
131  & \textit{Pongo pygmaeus} & NC\_001646 & S & 2.8500 & 4.7160 \\
132  & \textit{Pontoporia blainvillei} & NC\_005277 & S & 2.2833 & 4.6075 \\
133  & \textit{Potorous tridactylus} & NC\_006524 & S & 2.2405 & 2.9894 \\
134  & \textit{Presbytis melalophos} & NC\_008217 & S & 2.3802 & 3.7993 \\
135  & \textit{Procavia capensis} & NC\_004919 & S & 2.2494 & 3.4771 \\
136  & \textit{Procyon lotor} & NC\_009126 & S & 2.4014 & 3.7324 \\
137  & \textit{Pseudocheirus peregrinus} & NC\_006519 & S & 2.0835 & 2.9619 \\
138  & \textit{Pteropus dasymallus} & NC\_002612 & S & 2.4594 & 2.6675 \\
139  & \textit{Pteropus scapulatus} & NC\_002619 & S & 2.2778 & 2.6021 \\
140  & \textit{Pygathrix nemaeus} & NC\_008220 & S & 2.4942 & 3.9196 \\
141  & \textit{Pygathrix roxellana} & NC\_008218 & S & 2.5490 & 4.3617 \\
142  & \textit{Rangifer tarandus} & NC\_007703 & S & 2.4156 & 5.1335 \\
143  & \textit{Rattus norvegicus} & NC\_001665 & S & 1.7782 & 2.5551 \\
144  & \textit{Rattus rattus} & NC\_012374 & L & 1.7024 & 2.2788 \\
145  & \textit{Rhinoceros unicornis} & NC\_001779 & S & 2.7177 & 6.2304 \\
146  & \textit{Rousettus aegyptiacus} & NC\_007393 & L & 2.4390 & 2.0969 \\
147  & \textit{Sciurus vulgaris} & NC\_002369 & S & 2.4390 & 2.6684 \\
148  & \textit{Semnopithecus entellus} & NC\_008215 & S & 2.5416 & 4.0618 \\
149  & \textit{Sminthopsis crassicaudata} & NC\_007631 & L & 1.7782 & 1.2148 \\
150  & \textit{Sorex unguiculatus} & NC\_005435 & L & 1.2553 & 1.1492 \\
151  & \textit{Spilogale putorius} & NC\_010497 & S & 2.1004 & 2.7868 \\
152  & \textit{Sus scrofa} & NC\_000845 & S & 2.5105 & 4.9832 \\
153  & \textit{Tachyglossus aculeatus} & NC\_003321 & S & 2.7738 & 3.4354 \\
154  & \textit{Talpa europaea} & NC\_002391 & L & 1.9243 & 1.8865 \\
155  & \textit{Tamandua tetradactyla} & NC\_004032 & S & 2.3579 & 3.6537 \\
156  & \textit{Tarsipes rostratus} & NC\_006518 & L & 1.3802 & 1.0000 \\
157  & \textit{Tarsius bancanus} & NC\_002811 & L & 2.2914 & 1.9745 \\
158  & \textit{Tarsius syrichta} & NC\_012774 & L & 2.2833 & 2.0760 \\
159  & \textit{Thryonomys swinderianus} & NC\_002658 & S & 1.8116 & 3.7243 \\
160  & \textit{Trachypithecus obscurus} & NC\_006900 & S & 2.6094 & 3.8325 \\
161  & \textit{Tremarctos ornatus} & NC\_009969 & S & 2.6702 & 5.1263 \\
162  & \textit{Trichechus manatus} & NC\_010302 & S & 2.8274 & 5.6180 \\
163  & \textit{Trichosurus vulpecula} & NC\_003039 & S & 2.2806 & 3.4314 \\
164  & \textit{Tupaia belangeri} & NC\_002521 & L & 2.1245 & 2.3010 \\
165  & \textit{Uncia uncia} & NC\_010638 & S & 2.4055 & 4.6812 \\
166  & \textit{Urotrichus talpoides} & NC\_005034 & L & 1.6232 & 1.3096 \\
167  & \textit{Ursus americanus} & NC\_003426 & S & 2.6107 & 5.1761 \\
168  & \textit{Ursus arctos} & NC\_003427 & S & 2.6812 & 5.3222 \\
169  & \textit{Ursus maritimus} & NC\_003428 & S & 2.7207 & 5.5911 \\
170  & \textit{Ursus thibetanus} & NC\_009971 & S & 2.6725 & 4.9482 \\
171  & \textit{Vombatus ursinus} & NC\_003322 & S & 2.5563 & 4.4150 \\
172  & \textit{Vulpes vulpes} & NC\_008434 & S & 2.4076 & 3.7243 \\
173  & \textit{Zaglossus bruijni} & NC\_006364 & S & 2.6941 & 4.0128 \\
%174  & \textit{Zalophus californianus} & NC\_008416 & S & 2.6318 & 5.2788 \\
	\hline
\end{longtable}

\twocolumn
Los an�lisis sobre las secuencias de DNA mitocondrial realizados en el cap�tulo \ref{cap:Met}, y cuyos resultados se muestran en la figura \ref{Fig1.3}, se realizaron de igual forma sobre un conjunto de prote�nas codificadas por nDNA que forman parte de la cadena respiratoria mitocondrial. En la tabla \ref{TablaMetnDNA} se muestran los identificadores de los genes que codifican dichas prote�nas. Los resultados obtenidos tras estos an�lisis han sido incluidos en el ap�ndice \ref{anexo:MetnDNA}.

% Tabla nDNA Met
\topcaption{\label{TablaMetnDNA} \small{Cat�logo de genes analizados codificados por nDNA. 38 del complejo I, 9 del complejo III y 14 del complejo IV. Estas secuencias fueron obtenidas de la base de datos KEGG (\textit{Kyoto Encyclopedia of Genes and Genomes}, \url{http://www.genome.jp/kegg}).}}
\begin{center}
\tablefirsthead{\hline
	no. & Gen & Id. \\
	\hline}
	\tablehead{%
	\hline
	no. & Gen & Id. \\
	\hline
	}
    \tabletail{}
    \tablelasttail{}
\begin{supertabular}{l l c}
\multicolumn{3}{c}{\textbf{Complejo I}} \\
\hline
1 & NDUFS1 & K03934 \\ 
2 & NDUFS2 & K03935 \\ 
3 & NDUFS3 & K03936 \\ 
4 & NDUFS4 & K03937 \\ 
5 & NDUFS5 & K03938 \\ 
6 & NDUFS6 & K03939 \\ 
7 & NDUFS7 & K03940 \\ 
8 & NDUFS8 & K03941 \\ 
9 & NDUFV1 & K03942 \\ 
10 & NDUFV2 & K03943 \\ 
11 & NDUFV3 & K03944 \\ 
12 & NDUFA1 & K03945 \\ 
13 & NDUFA2 & K03946 \\ 
14 & NDUFA3 & K03947 \\ 
15 & NDUFA4 & K03948 \\ 
16 & NDUFA5 & K03949 \\ 
17 & NDUFA6 & K03950 \\ 
18 & NDUFA7 & K03951 \\ 
19 & NDUFA8 & K03952 \\ 
20 & NDUFA9 & K03953 \\ 
21 & NDUFA10 & K03954 \\ 
22 & NDUFAB1 & K03955 \\ 
23 & NDUFA11 & K03956 \\ 
24 & NDUFA12 & K11352 \\ 
25 & NDUFA13 & K11353 \\ 
26 & NDUFB1 & K03957 \\ 
27 & NDUFB2 & K03958 \\ 
28 & NDUFB3 & K03959 \\ 
29 & NDUFB4 & K03960 \\ 
30 & NDUFB5 & K03961 \\ 
31 & NDUFB6 & K03962 \\ 
32 & NDUFB7 & K03963 \\ 
33 & NDUFB8 & K03964 \\ 
34 & NDUFB9 & K03965 \\ 
35 & NDUFB10 & K03966 \\ 
36 & NDUFB11 & K11351 \\ 
37 & NDUFC1 & K03967 \\ 
38 & NDUFC2 & K03968 \\
& & \\
\hline
\multicolumn{3}{c}{\textbf{Complejo III}} \\
\hline
no. & Gen & Id. \\
\hline
1 & UQCRFS1 & K00411 \\ 
2 & CYC1 & K00413 \\ 
3 & QCR1 & K00414 \\ 
4 & QCR2 & K00415 \\ 
5 & QCR6 & K00416 \\ 
6 & QCR7 & K00417 \\ 
7 & QCR8 & K00418 \\ 
8 & QCR9 & K00419 \\ 
9 & QCR10 & K00420 \\ 
& & \\
\hline
\multicolumn{3}{c}{\textbf{Complejo IV}} \\
\hline
no. & Gen & Id. \\
\hline
1 & COX10 & K02257 \\ 
2 & COX4 & K02263 \\ 
3 & COX5A & K02264 \\ 
4 & COX5B & K02265 \\ 
5 & COX6A & K02266 \\ 
6 & COX6B & K02267 \\ 
7 & COX6C & K02268 \\ 
8 & COX7 & K02269 \\ 
9 & COX7B & K02269 \\ 
10 & COX7C & K02272 \\ 
11 & COX8 & K02273 \\ 
12 & COX11 & K02258 \\ 
13 & COX15 & K02259 \\ 
14 & COX17 & K0226 \\
\hline
\end{supertabular}
\end{center}
% End nDNA Met

\pagebreak
\twocolumn
  \begin{@twocolumnfalse}
  \section{Datos del cap�tulo 2}
% Tabla Data 2
%\LTcapwidth=\textwidth
\topcaption{\label{TablaDatos2} \small{Cat�logo de las 311 especies de las que se recopilaron las secuencias gen�ticas (cDNA) de los genes citocromo b y COX 1. Tras llevar a cabo los alineamientos y la posterior depuraci�n de los datos, la muestra final que se analiz� fue de 231 especies. En la tercera columna se indica el \textit{accession number} del genoma mitocondrial de la especie.}}	
  \end{@twocolumnfalse}

\tablefirsthead{\hline
	no. & Especie & \emph{Accession number} \\
	\hline}
	\tablehead{%
	\hline
	no. & Especie & \emph{Accession number} \\
	\hline
	}
    \tabletail{}
    \tablelasttail{}
\begin{supertabular}{m{.5cm} m{4cm} m{1.5cm}}
1 & \textit{Acinonyx jubatus} & NC\_005212 \\
2 & \textit{Ailuropoda melanoleuca} & NC\_009492 \\
3 & \textit{Ailurus fulgens} & NC\_011124 \\
4 & \textit{Ammotragus lervia} & NC\_009510 \\
5 & \textit{Anomalurus sp} & NC\_009056 \\
6 & \textit{Arctocephalus forsteri} & NC\_004023 \\
7 & \textit{Arctocephalus pusillus} & NC\_008417 \\
8 & \textit{Artibeus jamaicensis} & NC\_002009 \\
9 & \textit{Chrysochloris asiatica} & NC\_004920 \\
10 & \textit{Eremitalpa granti} & NC\_010304 \\
11 & \textit{Ammotragus lervia} & NC\_009510 \\
12 & \textit{Bos indicus} & NC\_005971 \\
13 & \textit{Bos taurus} & NC\_006853 \\
14 & \textit{Camelus bactrianus} & NC\_009628 \\
15 & \textit{Camelus dromedarius} & NC\_009849 \\
16 & \textit{Capra hircus} & NC\_005044 \\
17 & \textit{Cervus elaphus} & NC\_007704 \\
18 & \textit{Cervus unicolor} & NC\_008414 \\
19 & \textit{Muntiacus muntjak} & NC\_004563 \\
20 & \textit{Muntiacus reevesi} & NC\_004069 \\
21 & \textit{Ammotragus lervia} & NC\_009510 \\
22 & \textit{Phacochoerus africanus} & NC\_008830 \\
23 & \textit{Sus scrofa} & NC\_000845 \\
24 & \textit{Balaena mysticetus} & NC\_005268 \\
25 & \textit{Balaenoptera acutorostrata} & NC\_005271 \\
26 & \textit{Balaenoptera bonaerensis} & NC\_006926 \\
27 & \textit{Balaenoptera borealis} & NC\_006929 \\
28 & \textit{Balaenoptera brydei} & NC\_006928 \\
29 & \textit{Balaenoptera edeni} & NC\_007938 \\
30 & \textit{Balaenoptera musculus} & NC\_001601 \\
31 & \textit{Balaenoptera physalus} & NC\_001321 \\
32 & \textit{Berardius bairdii} & NC\_005274 \\
33 & \textit{Bos grunniens} & NC\_006380 \\
34 & \textit{Bos indicus} & NC\_005971 \\
35 & \textit{Bos taurus} & NC\_006853 \\
36 & \textit{Bradypus tridactylus} & NC\_006923 \\
37 & \textit{Bubalus bubalis} & NC\_006295 \\
38 & \textit{Callorhinus ursinus} & NC\_008415 \\
39 & \textit{Camelus bactrianus} & NC\_009628 \\
40 & \textit{Camelus dromedarius} & NC\_009849 \\
41 & \textit{Canis latrans} & NC\_008093 \\
42 & \textit{Canis lupus} & NC\_008092 \\
43 & \textit{Caperea marginata} & NC\_005269 \\
44 & \textit{Capra hircus} & NC\_005044 \\
45 & \textit{Arctocephalus forsteri} & NC\_004023 \\
46 & \textit{Canis latrans} & NC\_008093 \\
47 & \textit{Canis lupus} & NC\_008092 \\
48 & \textit{Enhydra lutris} & NC\_009692 \\
49 & \textit{Eumetopias jubatus} & NC\_004030 \\
50 & \textit{Halichoerus grypus} & NC\_001602 \\
51 & \textit{Leptonychotes weddellii} & NC\_008424 \\
52 & \textit{Lobodon carcinophaga} & NC\_008423 \\
53 & \textit{Lutra lutra} & NC\_011358 \\
54 & \textit{Martes melampus} & NC\_009678 \\
55 & \textit{Martes zibellina} & NC\_011579 \\
56 & \textit{Panthera pardus} & NC\_010641 \\
57 & \textit{Panthera tigris} & NC\_010642 \\
58 & \textit{Phoca fasciata} & NC\_008428 \\
59 & \textit{Phoca groenlandica} & NC\_008429 \\
60 & \textit{Phoca largha} & NC\_008430 \\
61 & \textit{Phoca sibirica} & NC\_008432 \\
62 & \textit{Phoca vitulina} & NC\_001325 \\
63 & \textit{Phocarctos hookeri} & NC\_008418 \\
64 & \textit{Ursus arctos} & NC\_003427 \\
65 & \textit{Ursus maritimus} & NC\_003428 \\
66 & \textit{Zalophus californianus} & NC\_008416 \\
67 & \textit{Cavia porcellu} & NC\_000884 \\
68 & \textit{Cebus albifrons} & NC\_002763 \\
69 & \textit{Ceratotherium simum} & NC\_001808 \\
70 & \textit{Cervus elaphus} & NC\_007704 \\
71 & \textit{Cervus nippon centralis} & NC\_006993 \\
72 & \textit{Cervus unicolor} & NC\_008414 \\
73 & \textit{Balaenoptera acutorostrata} & NC\_005271 \\
74 & \textit{Balaenoptera bonaerensis} & NC\_006926 \\
75 & \textit{Balaenoptera brydei} & NC\_006928 \\
76 & \textit{Balaenoptera edeni} & NC\_007938 \\
77 & \textit{Balaenoptera physalus} & NC\_001321 \\
78 & \textit{Berardius bairdii} & NC\_005274 \\
79 & \textit{Eubalaena australis} & NC\_006930 \\
80 & \textit{Eubalaena japonica} & NC\_006931 \\
81 & \textit{Hyperoodon ampullatus} & NC\_005273 \\
82 & \textit{Kogia breviceps} & NC\_005272 \\
83 & \textit{Megaptera novaeangliae} & NC\_006927 \\
84 & \textit{Monodon monoceros} & NC\_005279 \\
85 & \textit{Phocoena phocoena} & NC\_005280 \\
86 & \textit{Physeter catodon} & NC\_002503 \\
87 & \textit{Chalinolobus tuberculatus} & NC\_002626 \\
88 & \textit{Artibeus jamaicensis} & NC\_002009 \\
89 & \textit{Chalinolobus tuberculatus} & NC\_002626 \\
90 & \textit{Mystacina tuberculata} & NC\_006925 \\
91 & \textit{Pipistrellus abramus} & NC\_005436 \\
92 & \textit{Pteropus dasymallus} & NC\_002612 \\
93 & \textit{Pteropus scapulatus} & NC\_002619 \\
94 & \textit{Rhinolophus monoceros} & NC\_005433 \\
95 & \textit{Rhinolophus pumilus} & NC\_005434 \\
96 & \textit{Chlorocebus aethiops} & NC\_007009 \\
97 & \textit{Choloepus didactylus} & NC\_006924 \\
98 & \textit{Chrysochloris asiatica} & NC\_004920 \\
99 & \textit{Colobus guereza} & NC\_006901 \\
100 & \textit{Cricetulus griseus} & NC\_007936 \\
101 & \textit{Crocidura russula} & NC\_006893 \\
102 & \textit{Cystophora cristata} & NC\_008427 \\
103 & \textit{Dactylopsila trivirgata} & NC\_008134 \\
104 & \textit{Dasypus novemcinctus} & NC\_001821 \\
105 & \textit{Dasyurus hallucatus} & NC\_007630 \\
106 & \textit{Myrmecobius fasciatus} & NC\_011949 \\
107 & \textit{Phascogale tapoatafa} & NC\_006523 \\
108 & \textit{Sminthopsis douglasi} & NC\_006517 \\
109 & \textit{Dasyurus hallucatus} & NC\_007630 \\
110 & \textit{Daubentonia madagascariensis} & NC\_010299 \\
111 & \textit{Dendrohyrax dorsalis} & NC\_010301 \\
112 & \textit{Metachirus nudicaudatus} & NC\_006516 \\
113 & \textit{Thylamys elegans} & NC\_005825 \\
114 & \textit{Didelphis virginiana} & NC\_001610 \\
115 & \textit{Dactylopsila trivirgata} & NC\_008134 \\
116 & \textit{Petaurus breviceps} & NC\_008135 \\
117 & \textit{Phalanger interpositus} & NC\_008137 \\
118 & \textit{Phascolarctos cinereus} & NC\_008133 \\
119 & \textit{Pseudocheirus peregrinus} & NC\_006519 \\
120 & \textit{Tarsipes rostratus} & NC\_006518 \\
121 & \textit{Trichosurus vulpecula} & NC\_003039 \\
122 & \textit{Vombatus ursinus} & NC\_003322 \\
123 & \textit{Distoechurus pennatus} & NC\_008145 \\
124 & \textit{Dugong dugon} & NC\_003314 \\
125 & \textit{Echinops telfairi} & NC\_002631 \\
126 & \textit{Echymipera rufescens australis} & NC\_007632 \\
127 & \textit{Elaphodus cephalophus} & NC\_008749 \\
128 & \textit{Elephantulus sp} & NC\_004921 \\
129 & \textit{Elephas maximus} & NC\_005129 \\
130 & \textit{Enhydra lutris} & NC\_009692 \\
131 & \textit{Equus asinus} & NC\_001788 \\
132 & \textit{Equus caballus} & NC\_001640 \\
133 & \textit{Eremitalpa granti} & NC\_010304 \\
134 & \textit{Erinaceus europaeus} & NC\_002080 \\
135 & \textit{Hemiechinus auritus} & NC\_005033 \\
136 & \textit{Erinaceus europaeus} & NC\_002080 \\
137 & \textit{Eschrichtius robustus} & NC\_005270 \\
138 & \textit{Eubalaena australis} & NC\_006930 \\
139 & \textit{Eubalaena japonica} & NC\_006931 \\
140 & \textit{Eulemur fulvus fulvus} & NC\_012766 \\
141 & \textit{Eulemur fulvus mayottensis} & NC\_012769 \\
142 & \textit{Eulemur mongoz}  & NC\_010300 \\
143 & \textit{Eumetopias jubatus} & NC\_004030 \\
144 & \textit{Felis catus} & NC\_001700 \\
145 & \textit{Galemys pyrenaicus} & NC\_008156 \\
146 & \textit{Glis glis} & NC\_001892 \\
147 & \textit{Gorilla gorilla} & NC\_001645 \\
148 & \textit{Gulo gulo} & NC\_009685 \\
149 & \textit{Halichoerus grypus} & NC\_001602 \\
150 & \textit{Helarctos malayanus} & NC\_009968 \\
151 & \textit{Hemiechinus auritus} & NC\_005033 \\
152 & \textit{Herpestes javanicus} & NC\_006835 \\
153 & \textit{Hippopotamus amphibius} & NC\_000889 \\
154 & \textit{Homo sapiens} & NC\_012920 \\
155 & \textit{Hydrurga leptonyx} & NC\_008425 \\
156 & \textit{Hylobates lar} & NC\_002082 \\
157 & \textit{Hyperoodon ampullatus} & NC\_005273 \\
158 & \textit{Dendrohyrax dorsalis} & NC\_010301 \\
159 & \textit{Procavia capensis} & NC\_004919 \\
160 & \textit{Inia geoffrensis} & NC\_005276 \\
161 & \textit{Isoodon macrourus} & NC\_002746 \\
162 & \textit{Jaculus jaculus} & NC\_005314 \\
163 & \textit{Kogia breviceps} & NC\_005272 \\
164 & \textit{Ochotona collaris} & NC\_003033 \\
165 & \textit{Ochotona princeps} & NC\_005358 \\
166 & \textit{Lama pacos} & NC\_002504 \\
167 & \textit{Lemur catta} & NC\_004025 \\
168 & \textit{Leptonychotes weddellii} & NC\_008424 \\
169 & \textit{Lepus europaeus} & NC\_004028 \\
170 & \textit{Lipotes vexillifer} & NC\_007629 \\
171 & \textit{Lobodon carcinophaga} & NC\_008423 \\
172 & \textit{Loxodonta africana} & NC\_000934 \\
173 & \textit{Lutra lutra} & NC\_011358 \\
174 & \textit{Macaca mulatta} & NC\_005943 \\
175 & \textit{Macaca sylvanus} & NC\_002764 \\
176 & \textit{Macropus robustus} & NC\_001794 \\
177 & \textit{Macroscelides proboscideus} & NC\_004026 \\
178 & \textit{Macrotis lagotis} & NC\_006520 \\
179 & \textit{Mammuthus primigenius} & NC\_007596 \\
180 & \textit{Martes flavigula} & NC\_012141 \\
181 & \textit{Martes melampus} & NC\_009678 \\
182 & \textit{Martes zibellina} & NC\_011579 \\
183 & \textit{Megaptera novaeangliae} & NC\_006927 \\
184 & \textit{Meles meles anakuma} & NC\_009677 \\
185 & \textit{Meles meles} & NC\_011125 \\
186 & \textit{Melursus ursinus} & NC\_009970 \\
187 & \textit{Metachirus nudicaudatus} & NC\_006516 \\
188 & \textit{Microtus kikuchii} & NC\_003041 \\
189 & \textit{Microtus rossiaemeridionalis} & NC\_008064 \\
190 & \textit{Mirounga leonina} & NC\_008422 \\
191 & \textit{Mogera wogura} & NC\_005035 \\
192 & \textit{Monachus schauinslandi} & NC\_008421 \\
193 & \textit{Monodelphis domestica} & NC\_006299 \\
194 & \textit{Monodon monoceros} & NC\_005279 \\
195 & \textit{Ornithorhynchus anatinus} & NC\_000891 \\
196 & \textit{Tachyglossus aculeatus} & NC\_003321 \\
197 & \textit{Muntiacus muntjak} & NC\_004563 \\
198 & \textit{Muntiacus reevesi} & NC\_004069 \\
199 & \textit{Mus musculus domesticus} & NC\_006914 \\
200 & \textit{Myrmecobius fasciatus} & NC\_011949 \\
201 & \textit{Mystacina tuberculata} & NC\_006925 \\
202 & \textit{Nannospalax ehrenbergi} & NC\_005315 \\
203 & \textit{Nasalis larvatus} & NC\_008216 \\
204 & \textit{Neofelis nebulosa} & NC\_008450 \\
205 & \textit{Neophoca cinerea} & NC\_008419 \\
206 & \textit{Nycticebus coucang} & NC\_002765 \\
207 & \textit{Ochotona collaris} & NC\_003033 \\
208 & \textit{Ochotona princeps} & NC\_005358 \\
209 & \textit{Ornithorhynchus anatinus} & NC\_000891 \\
210 & \textit{Orycteropus afer} & NC\_002078 \\
211 & \textit{Oryctolagus cuniculus} & NC\_001913 \\
212 & \textit{Ovis aries} & NC\_001941 \\
213 & \textit{Pan paniscus} & NC\_001644 \\
214 & \textit{Pan troglodytes} & NC\_001643 \\
215 & \textit{Panthera pardus} & NC\_010641 \\
216 & \textit{Panthera tigris} & NC\_010642 \\
217 & \textit{Papio hamadryas} & NC\_001992 \\
218 & \textit{Isoodon macrourus} & NC\_002746 \\
219 & \textit{Macrotis lagotis} & NC\_006520 \\
220 & \textit{Pecari tajacu} & NC\_012103 \\
221 & \textit{Perameles gunnii} & NC\_006521 \\
222 & \textit{Ceratotherium simum} & NC\_001808 \\
223 & \textit{Rhinoceros unicornis} & NC\_001779 \\
224 & \textit{Petaurus breviceps} & NC\_008135 \\
225 & \textit{Phacochoerus africanus} & NC\_008830 \\
226 & \textit{Phalanger interpositus} & NC\_008137 \\
227 & \textit{Phascogale tapoatafa} & NC\_006523 \\
228 & \textit{Phascolarctos cinereus} & NC\_008133 \\
229 & \textit{Phoca fasciata} & NC\_008428 \\
230 & \textit{Phoca groenlandica} & NC\_008429 \\
231 & \textit{Phoca hispida} & NC\_008433 \\
232 & \textit{Phoca largha} & NC\_008430 \\
233 & \textit{Phoca sibirica} & NC\_008432 \\
234 & \textit{Phoca vitulina} & NC\_001325 \\
235 & \textit{Phocarctos hookeri} & NC\_008418 \\
236 & \textit{Phocoena phocoena} & NC\_005280 \\
237 & \textit{Physeter catodon} & NC\_002503 \\
238 & \textit{Bradypus tridactylus} & NC\_006923 \\
239 & \textit{Choloepus didactylus} & NC\_006924 \\
240 & \textit{Pipistrellus abramus} & NC\_005436 \\
241 & \textit{Pongo abelii} & NC\_002083 \\
242 & \textit{Pongo pygmaeus} & NC\_001646 \\
243 & \textit{Pontoporia blainvillei} & NC\_005277 \\
244 & \textit{Potorous tridactylus} & NC\_006524 \\
245 & \textit{Presbytis melalophos} & NC\_008217 \\
246 & \textit{Colobus guereza} & NC\_006901 \\
247 & \textit{Eulemur fulvus fulvus} & NC\_012766 \\
248 & \textit{Eulemur fulvus mayottensis} & NC\_012769 \\
249 & \textit{Macaca sylvanus} & NC\_002764 \\
250 & \textit{Nasalis larvatus} & NC\_008216 \\
251 & \textit{Papio hamadryas} & NC\_001992 \\
252 & \textit{Pongo abelii} & NC\_002083 \\
253 & \textit{Pongo pygmaeus} & NC\_001646 \\
254 & \textit{Procolobus badius} & NC\_008219 \\
255 & \textit{Pygathrix roxellana} & NC\_008218 \\
256 & \textit{Tarsius bancanus} & NC\_002811 \\
257 & \textit{Tarsius syrichta} & NC\_012774 \\
258 & \textit{Elephas maximus} & NC\_005129 \\
259 & \textit{Mammuthus primigenius} & NC\_007596 \\
260 & \textit{Procavia capensis} & NC\_004919 \\
261 & \textit{Procolobus badius} & NC\_008219 \\
262 & \textit{Procyon lotor} & NC\_009126 \\
263 & \textit{Pseudocheirus peregrinus} & NC\_006519 \\
264 & \textit{Pteropus dasymallus} & NC\_002612 \\
265 & \textit{Pteropus scapulatus} & NC\_002619 \\
266 & \textit{Pygathrix nemaeus} & NC\_008220 \\
267 & \textit{Pygathrix roxellana} & NC\_008218 \\
268 & \textit{Rangifer tarandus} & NC\_007703 \\
269 & \textit{Rattus norvegicus} & NC\_001665 \\
270 & \textit{Rattus rattus} & NC\_012374 \\
271 & \textit{Rhinoceros unicornis} & NC\_001779 \\
272 & \textit{Rhinolophus monoceros} & NC\_005433 \\
273 & \textit{Rhinolophus pumilus} & NC\_005434 \\
274 & \textit{Microtus kikuchii} & NC\_003041 \\
275 & \textit{Microtus rossiaemeridionalis} & NC\_008064 \\
276 & \textit{Rattus norvegicus} & NC\_001665 \\
277 & \textit{Rattus rattus} & NC\_012374 \\
278 & \textit{Sciurus vulgaris} & NC\_002369 \\
279 & \textit{Thryonomys swinderianus} & NC\_002658 \\
280 & \textit{Sciurus vulgaris} & NC\_002369 \\
281 & \textit{Semnopithecus entellus} & NC\_008215 \\
282 & \textit{Sminthopsis crassicaudata} & NC\_007631 \\
283 & \textit{Sminthopsis douglasi} & NC\_006517 \\
284 & \textit{Sorex unguiculatus} & NC\_005435 \\
285 & \textit{Galemys pyrenaicus} & NC\_008156 \\
286 & \textit{Urotrichus talpoides} & NC\_005034 \\
287 & \textit{Spilogale putorius} & NC\_010497 \\
288 & \textit{Sus scrofa} & NC\_000845 \\
289 & \textit{Tachyglossus aculeatus} & NC\_003321 \\
290 & \textit{Talpa europaea} & NC\_002391 \\
291 & \textit{Tamandua tetradactyla} & NC\_004032 \\
292 & \textit{Tarsipes rostratus} & NC\_006518 \\
293 & \textit{Tarsius bancanus} & NC\_002811 \\
294 & \textit{Tarsius syrichta} & NC\_012774 \\
295 & \textit{Thryonomys swinderianus} & NC\_002658 \\
296 & \textit{Thylamys elegans} & NC\_005825 \\
297 & \textit{Trachypithecus obscurus} & NC\_006900 \\
298 & \textit{Tremarctos ornatus} & NC\_009969 \\
299 & \textit{Trichechus manatus} & NC\_010302 \\
300 & \textit{Trichosurus vulpecula} & NC\_003039 \\
301 & \textit{Tupaia belangeri} & NC\_002521 \\
302 & \textit{Uncia uncia} & NC\_010638 \\
303 & \textit{Urotrichus talpoides} & NC\_005034 \\
304 & \textit{Ursus americanus} & NC\_003426 \\
305 & \textit{Ursus arctos} & NC\_003427 \\
306 & \textit{Ursus maritimus} & NC\_003428 \\
307 & \textit{Ursus thibetanus} & NC\_009971 \\
308 & \textit{Vombatus ursinus} & NC\_003322 \\
309 & \textit{Vulpes vulpes} & NC\_008434 \\
310 & \textit{Zaglossus bruijni} & NC\_006364 \\
311 & \textit{Zalophus californianus} & NC\_008416 \\
	\hline
\end{supertabular}
% End Tabla Data 2
\pagebreak

% Tabla Data 3
\twocolumn
  \begin{@twocolumnfalse}
  \section{Datos del cap�tulo 3}
\topcaption{\label{TablaDatos3} \small{Cat�logo de las 371 especies de mam�feros para las que se recopilaron las secuencias gen�ticas de las subunidades del complejo citocromo c oxidasa (COX), codificadas por mtDNA (COX 1, COX 2 y COX 3). En la tercera columna se indica el \textit{accession number} del genoma mitocondrial de la especie.}}	
  \end{@twocolumnfalse}

\tablefirsthead{\hline
	no. & Especie & \emph{Accession number} \\
	\hline}
	\tablehead{%
	\hline
	no. & Especie & \emph{Accession number} \\
	\hline
	}
    \tabletail{}
    \tablelasttail{}\begin{supertabular}{m{.5cm} m{4cm} m{1.5cm}}
1 & \textit{Bos taurus} & NC\_006853 \\
2 & \textit{Acinonyx jubatus} & NC\_005212 \\
3 & \textit{Ailuropoda melanoleuca} & NC\_009492 \\
4 & \textit{Ailurus fulgens} & NC\_011124 \\
5 & \textit{Ailurus fulgens styani} & NC\_009691 \\
6 & \textit{Ammotragus lervia} & NC\_009510 \\
7 & \textit{Anomalurus sp} & NC\_009056 \\
8 & \textit{Antilope cervicapra} & NC\_012098 \\
9 & \textit{Aotus azarai azarai} & NC\_018115 \\
10 & \textit{Aotus nancymaae} & NC\_018116 \\
11 & \textit{Apodemus agrarius} & NC\_016428 \\
12 & \textit{Apodemus chejuensis} & NC\_016662 \\
13 & \textit{Apodemus chevrieri} & NC\_017599 \\
14 & \textit{Apodemus peninsulae} & NC\_016060 \\
15 & \textit{Arctocephalus pusillus} & NC\_008417 \\
16 & \textit{Arctocephalus townsendi} & NC\_008420 \\
17 & \textit{Arctodus simus} & NC\_011116 \\
18 & \textit{Artibeus jamaicensis} & NC\_002009 \\
19 & \textit{Artibeus lituratus} & NC\_016871 \\
20 & \textit{Artocephalus forsteri} & NC\_004023 \\
21 & \textit{Balaena mysticetus} & NC\_005268 \\
22 & \textit{Balaenoptera acutorostrata} & NC\_005271 \\
23 & \textit{Balaenoptera bonaerensis} & NC\_006926 \\
24 & \textit{Balaenoptera borealis} & NC\_006929 \\
25 & \textit{Balaenoptera brydei} & NC\_006928 \\
26 & \textit{Balaenoptera edeni} & NC\_007938 \\
27 & \textit{Balaenoptera musculus} & NC\_001601 \\
28 & \textit{Balaenoptera omurai} & NC\_007937 \\
29 & \textit{Balaenoptera physalus} & NC\_001321 \\
30 & \textit{Berardius bairdii} & NC\_005274 \\
31 & \textit{Bison bison} & NC\_012346 \\
32 & \textit{Bison bonasus} & NC\_014044 \\
33 & \textit{Bos grunniens} & NC\_006380 \\
34 & \textit{Bos indicus} & NC\_005971 \\
35 & \textit{Bos javanicus} & NC\_012706 \\
36 & \textit{Bos primigenius} & NC\_013996 \\
37 & \textit{Bradypus tridactylus} & NC\_006923 \\
38 & \textit{Bubalus bubalis} & NC\_006295 \\
39 & \textit{Budorcas taxicolor} & NC\_013069 \\
40 & \textit{Caenolestes fuliginosus} & NC\_005828 \\
41 & \textit{Callorhinus ursinus} & NC\_008415 \\
42 & \textit{Camelus bactrianus} & NC\_009628 \\
43 & \textit{Camelus dromedarius} & NC\_009849 \\
44 & \textit{Camelus ferus} & NC\_009629 \\
45 & \textit{Canis latrans} & NC\_008093 \\
46 & \textit{Canis lupus} & NC\_008092 \\
47 & \textit{Canis lupus chanco} & NC\_010340 \\
48 & \textit{Canis lupus familiaris} & NC\_002008 \\
49 & \textit{Canis lupus laniger} & NC\_011218 \\
50 & \textit{Canis lupus lupus} & NC\_009686 \\
51 & \textit{Caperea marginata} & NC\_005269 \\
52 & \textit{Capra hircus} & NC\_005044 \\
53 & \textit{Capricornis crispus} & NC\_012096 \\
54 & \textit{Capricornis swinhoei} & NC\_010640 \\
55 & \textit{Castor canadensis} & NC\_015108 \\
56 & \textit{Castor fiber} & NC\_015072 \\
57 & \textit{Cavia porcellus} & NC\_000884 \\
58 & \textit{Cebus albifrons} & NC\_002763 \\
59 & \textit{Cebus apella} & NC\_016666 \\
60 & \textit{Ceratotherium simum} & NC\_001808 \\
61 & \textit{Cervus elaphus} & NC\_007704 \\
62 & \textit{Cervus elaphus songaricus} & NC\_014703 \\
63 & \textit{Cervus elaphus xanthopygus} & NC\_013836 \\
64 & \textit{Cervus elaphus yarkandensis} & NC\_013840 \\
65 & \textit{Cervus hortulorum} & NC\_013834 \\
66 & \textit{Cervus nippon centralis} & NC\_006993 \\
67 & \textit{Cervus nippon kopschi} & NC\_016178 \\
68 & \textit{Cervus nippon yakushimae} & NC\_007179 \\
69 & \textit{Cervus taiouanus} & NC\_008462 \\
70 & \textit{Cervus yesoensis} & NC\_006973 \\
71 & \textit{Chalinolobus tuberculatus} & NC\_002626 \\
72 & \textit{Chlorocebus aethiops} & NC\_007009 \\
73 & \textit{Chlorocebus pygerythrus} & NC\_009747 \\
74 & \textit{Chlorocebus sabaeus} & NC\_008066 \\
75 & \textit{Chlorocebus tantalus} & NC\_009748 \\
76 & \textit{Choloepus didactylus} & NC\_006924 \\
77 & \textit{Chrysochloris asiatica} & NC\_004920 \\
78 & \textit{Coelodonta antiquitatis} & NC\_012681 \\
79 & \textit{Colobus guereza} & NC\_006901 \\
80 & \textit{Cricetulus griseus} & NC\_007936 \\
81 & \textit{Crocidura russula} & NC\_006893 \\
82 & \textit{Cuon alpinus} & NC\_013445 \\
83 & \textit{Cystophora cristata} & NC\_008427 \\
84 & \textit{Dactylopsila trivirgata} & NC\_008134 \\
85 & \textit{Dasypus novemcinctus} & NC\_001821 \\
86 & \textit{Dasyurus hallucatus} & NC\_007630 \\
87 & \textit{Daubentonia madagascariensis} & NC\_010299 \\
88 & \textit{Delphinus capensis} & NC\_012061 \\
89 & \textit{Dendrohyrax dorsalis} & NC\_010301 \\
90 & \textit{Dicerorhinus sumatrensis} & NC\_012684 \\
91 & \textit{Diceros bicornis} & NC\_012682 \\
92 & \textit{Didelphis virginiana} & NC\_001610 \\
93 & \textit{Distoechurus pennatus} & NC\_008145 \\
94 & \textit{Dromiciops gliroides} & NC\_005826 \\
95 & \textit{Dugong dugon} & NC\_003314 \\
96 & \textit{Echinops telfairi} & NC\_002631 \\
97 & \textit{Echinosorex gymnura} & NC\_002808 \\
98 & \textit{Echymipera rufescens australis} & NC\_007632 \\
99 & \textit{Elaphodus cephalophus} & NC\_008749 \\
100 & \textit{Elephantulus sp} & NC\_004921 \\
101 & \textit{Elephas maximus} & NC\_005129 \\
102 & \textit{Enhydra lutris} & NC\_009692 \\
103 & \textit{Eospalax baileyi} & NC\_018098 \\
104 & \textit{Eothenomys chinensis} & NC\_013571 \\
105 & \textit{Episoriculus fumidus} & NC\_003040 \\
106 & \textit{Equus asinus} & NC\_001788 \\
107 & \textit{Equus caballus} & NC\_001640 \\
108 & \textit{Equus hemionus} & NC\_016061 \\
109 & \textit{Eremitalpa granti} & NC\_010304 \\
110 & \textit{Erignathus barbatus} & NC\_008426 \\
111 & \textit{Erinaceus europaeus} & NC\_002080 \\
112 & \textit{Eschrichtius robustus} & NC\_005270 \\
113 & \textit{Eubalaena australis} & NC\_006930 \\
114 & \textit{Eubalaena japonica} & NC\_006931 \\
115 & \textit{Eulemur fulvus fulvus} & NC\_012766 \\
116 & \textit{Eulemur fulvus mayottensis} & NC\_012769 \\
117 & \textit{Eulemur macaco macaco} & NC\_012771 \\
118 & \textit{Eulemur mongoz} & NC\_010300 \\
119 & \textit{Eumetopias jubatus} & NC\_004030 \\
120 & \textit{Felis catus} & NC\_001700 \\
121 & \textit{Galago senegalensis} & NC\_012761 \\
122 & \textit{Galemys pyrenaicus} & NC\_008156 \\
123 & \textit{Galeopterus variegatus} & NC\_004031 \\
124 & \textit{Giraffa camelopardalis angolensis} & NC\_012100 \\
125 & \textit{Glis glis} & NC\_001892 \\
126 & \textit{Gorilla gorilla} & NC\_001645 \\
127 & \textit{Gorilla gorilla gorilla} & NC\_011120 \\
128 & \textit{Grampus griseus} & NC\_012062 \\
129 & \textit{Gulo gulo} & NC\_009685 \\
130 & \textit{Halichoerus grypus} & NC\_001602 \\
131 & \textit{Helarctos malayanus} & NC\_009968 \\
132 & \textit{Hemiechinus auritus} & NC\_005033 \\
133 & \textit{Herpestes javanicus} & NC\_006835 \\
134 & \textit{Heterocephalus glaber} & NC\_015112 \\
135 & \textit{Hippopotamus amphibius} & NC\_000889 \\
136 & \textit{Homo sapiens} & NC\_012920 \\
137 & \textit{Homo sapiens neanderthalensis} & NC\_011137 \\
138 & \textit{Homo sp. Atai} & NC\_013993 \\
139 & \textit{Hydropotes inermis} & NC\_011821 \\
140 & \textit{Hydropotes inermis argyropus} & NC\_018032 \\
141 & \textit{Hydrurga leptonyx} & NC\_008425 \\
142 & \textit{Hylobates agilis} & NC\_014042 \\
143 & \textit{Hylobates lar} & NC\_002082 \\
144 & \textit{Hylobates pileatus} & NC\_014045 \\
145 & \textit{Hylomys suillus} & NC\_010298 \\
146 & \textit{Hyperoodon ampullatus} & NC\_005273 \\
147 & \textit{Inia geoffrensis} & NC\_005276 \\
148 & \textit{Isoodon macrourus} & NC\_002746 \\
149 & \textit{Jaculus jaculus} & NC\_005314 \\
150 & \textit{Kogia breviceps} & NC\_005272 \\
151 & \textit{Lagenorhynchus albirostris} & NC\_005278 \\
152 & \textit{Lagorchestes hirsutus} & NC\_008136 \\
153 & \textit{Lagostrophus fasciatus} & NC\_008447 \\
154 & \textit{Lama glama} & NC\_012102 \\
155 & \textit{Lama guanicoe} & NC\_011822 \\
156 & \textit{Lasiurus borealis} & NC\_016873 \\
157 & \textit{Leggadina lakedownensis} & NC\_014696 \\
158 & \textit{Lemur catta} & NC\_004025 \\
159 & \textit{Lepilemur hubbardorum} & NC\_014453 \\
160 & \textit{Leptonychotes weddellii} & NC\_008424 \\
161 & \textit{Lepus capensis} & NC\_015841 \\
162 & \textit{Lepus europaeus} & NC\_004028 \\
163 & \textit{Lipotes vexillifer} & NC\_007629 \\
164 & \textit{Lobodon carcinophaga} & NC\_008423 \\
165 & \textit{Loris tardigradus} & NC\_012763 \\
166 & \textit{Loxodonta africana} & NC\_000934 \\
167 & \textit{Lutra lutra} & NC\_011358 \\
168 & \textit{Lynx rufus} & NC\_014456 \\
169 & \textit{Macaca fascicularis} & NC\_012670 \\
170 & \textit{Macaca mulatta} & NC\_005943 \\
171 & \textit{Macaca sylvanus} & NC\_002764 \\
172 & \textit{Macaca thibetana} & NC\_011519 \\
173 & \textit{Macropus robustus} & NC\_001794 \\
174 & \textit{Macroscelides proboscideus} & NC\_004026 \\
175 & \textit{Macrotis lagotis} & NC\_006520 \\
176 & \textit{Mammut americanum} & NC\_009574 \\
177 & \textit{Mammuthus columbi} & NC\_015529 \\
178 & \textit{Mammuthus primigenius} & NC\_007596 \\
179 & \textit{Manis pentadactyla} & NC\_016008 \\
180 & \textit{Manis tetradactyla} & NC\_004027 \\
181 & \textit{Martes flavigula} & NC\_012141 \\
182 & \textit{Martes melampus} & NC\_009678 \\
183 & \textit{Martes zibellina} & NC\_011579 \\
184 & \textit{Megaptera novaeangliae} & NC\_006927 \\
185 & \textit{Meles anakuma} & NC\_009677 \\
186 & \textit{Meles meles} & NC\_011125 \\
187 & \textit{Melursus ursinus} & NC\_009970 \\
188 & \textit{Mesocricetus auratus} & NC\_013276 \\
189 & \textit{Metachirus nudicaudatus} & NC\_006516 \\
190 & \textit{Microtus fortis calamorum} & NC\_015243 \\
191 & \textit{Microtus fortis fortis} & NC\_015241 \\
192 & \textit{Microtus kikuchii} & NC\_003041 \\
193 & \textit{Microtus levis} & NC\_008064 \\
194 & \textit{Mirounga leonina} & NC\_008422 \\
195 & \textit{Mogera wogura} & NC\_005035 \\
196 & \textit{Monachus schauinslandi} & NC\_008421 \\
197 & \textit{Monodelphis domestica} & NC\_006299 \\
198 & \textit{Monodon monoceros} & NC\_005279 \\
199 & \textit{Moschus berezovskii} & NC\_012694 \\
200 & \textit{Moschus moschiferus} & NC\_013753 \\
201 & \textit{Muntiacus crinifrons} & NC\_004577 \\
202 & \textit{Muntiacus muntjak} & NC\_004563 \\
203 & \textit{Muntiacus reevesi} & NC\_004069 \\
204 & \textit{Muntiacus reevesi micrurus} & NC\_008491 \\
205 & \textit{Muntiacus vuguangensis} & NC\_016920 \\
206 & \textit{Mus musculus} & NC\_005089 \\
207 & \textit{Mus musculus castaneus} & NC\_012387 \\
208 & \textit{Mus musculus domesticus} & NC\_006914 \\
209 & \textit{Mus musculus molossinus} & NC\_006915 \\
210 & \textit{Mus musculus musculus} & NC\_010339 \\
211 & \textit{Mus terricolor} & NC\_010650 \\
212 & \textit{Myodes regulus} & NC\_016427 \\
213 & \textit{Myotis formosus} & NC\_015828 \\
214 & \textit{Myrmecobius fasciatus} & NC\_011949 \\
215 & \textit{Mystacina tuberculata} & NC\_006925 \\
216 & \textit{Naemorhedus caudatus} & NC\_013751 \\
217 & \textit{Nannospalax ehrenbergi} & NC\_005315 \\
218 & \textit{Nasalis larvatus} & NC\_008216 \\
219 & \textit{Neodon irene} & NC\_016055 \\
220 & \textit{Neofelis nebulosa} & NC\_008450 \\
221 & \textit{Neophoca cinerea} & NC\_008419 \\
222 & \textit{Nomascus siki} & NC\_014051 \\
223 & \textit{Notoryctes typhlops} & NC\_006522 \\
224 & \textit{Nyctereutes procyonoides} & NC\_013700 \\
225 & \textit{Nycticebus coucang} & NC\_002765 \\
226 & \textit{Ochotona collaris} & NC\_003033 \\
227 & \textit{Ochotona curzoniae} & NC\_011029 \\
228 & \textit{Ochotona princeps} & NC\_005358 \\
229 & \textit{Odobenus rosmarus rosmarus} & NC\_004029 \\
230 & \textit{Odocoileus virginianus} & NC\_015247 \\
231 & \textit{Orcinus orca} & NC\_014682 \\
232 & \textit{Ornithorhynchus anatinus} & NC\_000891 \\
233 & \textit{Orycteropus afer} & NC\_002078 \\
234 & \textit{Oryctolagus cuniculus} & NC\_001913 \\
235 & \textit{Oryx dammah} & NC\_016421 \\
236 & \textit{Oryx gazella} & NC\_016422 \\
237 & \textit{Otolemur crassicaudatus} & NC\_012762 \\
238 & \textit{Ovis aries} & NC\_001941 \\
239 & \textit{Ovis canadensis} & NC\_015889 \\
240 & \textit{Pan paniscus} & NC\_001644 \\
241 & \textit{Pan troglodytes} & NC\_001643 \\
242 & \textit{Panthera leo persica} & NC\_018053 \\
243 & \textit{Panthera pardus} & NC\_010641 \\
244 & \textit{Panthera tigris} & NC\_010642 \\
245 & \textit{Panthera tigris amoyensis} & NC\_014770 \\
246 & \textit{Pantholops hodgsonii} & NC\_007441 \\
247 & \textit{Papio hamadryas} & NC\_001992 \\
248 & \textit{Pecari tajacu} & NC\_012103 \\
249 & \textit{Perameles gunnii} & NC\_006521 \\
250 & \textit{Perodicticus potto} & NC\_012764 \\
251 & \textit{Petaurus breviceps} & NC\_008135 \\
252 & \textit{Phacochoerus africanus} & NC\_008830 \\
253 & \textit{Phalanger vestitus} & NC\_008137 \\
254 & \textit{Phascogale tapoatafa} & NC\_006523 \\
255 & \textit{Phascolarctos cinereus} & NC\_008133 \\
256 & \textit{Phoca fasciata} & NC\_008428 \\
257 & \textit{Phoca groenlandica} & NC\_008429 \\
258 & \textit{Phoca largha} & NC\_008430 \\
259 & \textit{Phoca vitulina} & NC\_001325 \\
260 & \textit{Phocarctos hookeri} & NC\_008418 \\
261 & \textit{Phocoena phocoena} & NC\_005280 \\
262 & \textit{Physeter catodon} & NC\_002503 \\
263 & \textit{Piliocolobus badius} & NC\_008219 \\
264 & \textit{Pipistrellus abramus} & NC\_005436 \\
265 & \textit{Platanista minor} & NC\_005275 \\
266 & \textit{Plecotus auritus} & NC\_015484 \\
267 & \textit{Plecotus rafinesquii} & NC\_016872 \\
268 & \textit{Pongo abelii} & NC\_002083 \\
269 & \textit{Pongo pygmaeus} & NC\_001646 \\
270 & \textit{Pontoporia blainvillei} & NC\_005277 \\
271 & \textit{Potorous tridactylus} & NC\_006524 \\
272 & \textit{Presbytis melalophos} & NC\_008217 \\
273 & \textit{Prionailurus bengalensis euptilurus} & NC\_016189 \\
274 & \textit{Procapra przewalskii} & NC\_014875 \\
275 & \textit{Procavia capensis} & NC\_004919 \\
276 & \textit{Procyon lotor} & NC\_009126 \\
277 & \textit{Proedromys liangshanensis} & NC\_013563 \\
278 & \textit{Propithecus coquereli} & NC\_011053 \\
279 & \textit{Przewalskium albirostris} & NC\_016707 \\
280 & \textit{Pseudocheirus peregrinus} & NC\_006519 \\
281 & \textit{Pseudois schaeferi} & NC\_016689 \\
282 & \textit{Pseudomys chapmani} & NC\_014698 \\
283 & \textit{Pteropus dasymallus} & NC\_002612 \\
284 & \textit{Pteropus scapulatus} & NC\_002619 \\
285 & \textit{Puma concolor} & NC\_016470 \\
286 & \textit{Pusa caspica} & NC\_008431 \\
287 & \textit{Pusa hispida} & NC\_008433 \\
288 & \textit{Pusa sibirica} & NC\_008432 \\
289 & \textit{Pygathrix cinera} 2 RL-2012 & NC\_018063 \\
290 & \textit{Pygathrix cinerea} 1 RL-2012 & NC\_018062 \\
291 & \textit{Pygathrix nemaeus} & NC\_008220 \\
292 & \textit{Pygathrix nigripes} & NC\_018061 \\
293 & \textit{Rangifer tarandus} & NC\_007703 \\
294 & \textit{Rattus exulans} & NC\_012389 \\
295 & \textit{Rattus fuscipes} & NC\_014867 \\
296 & \textit{Rattus leucopus} & NC\_014855 \\
297 & \textit{Rattus lutreolus} & NC\_014858 \\
298 & \textit{Rattus norvegicus} & NC\_001665 \\
299 & \textit{Rattus praetor} & NC\_012461 \\
300 & \textit{Rattus rattus} & NC\_012374 \\
301 & \textit{Rattus sordidus} & NC\_014871 \\
302 & \textit{Rattus tanezumi} & NC\_011638 \\
303 & \textit{Rattus tunneyi} & NC\_014861 \\
304 & \textit{Rattus villosissimus} & NC\_014864 \\
305 & \textit{Rhinoceros sondaicus} & NC\_012683 \\
306 & \textit{Rhinoceros unicornis} & NC\_001779 \\
307 & \textit{Rhinolophus ferrumequinum korai} & NC\_016191 \\
308 & \textit{Rhinolophus formosae} & NC\_011304 \\
309 & \textit{Rhinolophus monoceros} & NC\_005433 \\
310 & \textit{Rhinolophus pumilus} & NC\_005434 \\
311 & \textit{Rhinopithecus avunculus} & NC\_015485 \\
312 & \textit{Rhinopithecus bieti} & NC\_015486 \\
313 & \textit{Rhinopithecus bieti} 1 RL\_2012 & NC\_018058 \\
314 & \textit{Rhinopithecus bieti} 2 RL 2012 & NC\_018060 \\
315 & \textit{Rhinopithecus brelichi}  & NC\_018057 \\
316 & \textit{Rhinopithecus roxellana} & NC\_008218 \\
317 & \textit{Rhinopithecus strykeri} & NC\_018059 \\
318 & \textit{Rhyncholestes raphanurus} & NC\_005829 \\
319 & \textit{Rousettus aegyptiacus} & NC\_007393 \\
320 & \textit{Rucervus eldi} & NC\_014701 \\
321 & \textit{Rusa unicolor swinhoei} & NC\_008414 \\
322 & \textit{Saimiri boliviensis boliviensis} & NC\_018096 \\
323 & \textit{Saimiri sciureus} & NC\_012775 \\
324 & \textit{Sciurus vulgaris} & NC\_002369 \\
325 & \textit{Semnopithecus entellus} & NC\_008215 \\
326 & \textit{Sminthopsis crassicaudata} & NC\_007631 \\
327 & \textit{Sminthopsis douglasi} & NC\_006517 \\
328 & \textit{Sorex unguiculatus} & NC\_005435 \\
329 & \textit{Sousa chinensis} & NC\_012057 \\
330 & \textit{Spilogale putorius} & NC\_010497 \\
331 & \textit{Stenella attenuata} & NC\_012051 \\
332 & \textit{Stenella coeruleoalba} & NC\_012053 \\
333 & \textit{Sus scrofa} & NC\_000845 \\
334 & \textit{Sus scrofa domesticus} & NC\_012095 \\
335 & \textit{Sus scrofa taiwanensis} & NC\_014692 \\
336 & \textit{Symphalangus syndactylus} & NC\_014047 \\
337 & \textit{Tachyglossus aculeatus} & NC\_003321 \\
338 & \textit{Talpa europaea} & NC\_002391 \\
339 & \textit{Tamandua tetradactyla} & NC\_004032 \\
340 & \textit{Tarsipes rostratus} & NC\_006518 \\
341 & \textit{Tarsius bancanus} & NC\_002811 \\
342 & \textit{Tarsius syrichta} & NC\_012774 \\
343 & \textit{Thryonomys swinderianus} & NC\_002658 \\
344 & \textit{Thylacinus cynocephalus} & NC\_011944 \\
345 & \textit{Thylamys elegans} & NC\_005825 \\
346 & \textit{Trachypithecus obscurus} & NC\_006900 \\
347 & \textit{Tremarctos ornatus} & NC\_009969 \\
348 & \textit{Trichechus manatus} & NC\_010302 \\
349 & \textit{Trichosurus vulpecula} & NC\_003039 \\
350 & \textit{Tscherskia triton} & NC\_013068 \\
351 & \textit{Tupaia belangeri} & NC\_002521 \\
352 & \textit{Tursiops aduncus} & NC\_012058 \\
353 & \textit{Tursiops truncatus} & NC\_012059 \\
354 & \textit{Uncia uncia} & NC\_010638 \\
355 & \textit{Urotrichus talpoides} & NC\_005034 \\
356 & \textit{Ursus americanus} & NC\_003426 \\
357 & \textit{Ursus arctos} & NC\_003427 \\
358 & \textit{Ursus maritimus} & NC\_003428 \\
359 & \textit{Ursus spelaeus} & NC\_011112 \\
360 & \textit{Ursus thibetanus} & NC\_009971 \\
361 & \textit{Ursus thibetanus formosanus} & NC\_009331 \\
362 & \textit{Ursus thibetanus mupinensis} & NC\_008753 \\
363 & \textit{Ursus thibetanus thibetanus} & NC\_011118 \\
364 & \textit{Ursus thibetanus ussuricus} & NC\_011117 \\
365 & \textit{Varecia variegata variegata} & NC\_012773 \\
366 & \textit{Vicugna pacos} & NC\_002504 \\
367 & \textit{Vicugna vicugna} & NC\_013558 \\
368 & \textit{Vombatus ursinus} & NC\_003322 \\
369 & \textit{Vulpes vulpes} & NC\_008434 \\
370 & \textit{Zaglossus bruijni} & NC\_006364 \\
371 & \textit{Zalophus californianus} & NC\_008416 \\
	\hline
\end{supertabular}
% End Tabla Data 3
\pagebreak
\newpage
\restoregeometry
% Tabla Data 3 nDNA
\begin{landscape}
\begin{longtable}{l l c c c c c}
	\caption{\label{TablanDNA} \small{Cat�logo de secuencias de nDNA que codifican para las 10 subunidades pertenecientes al complejo citocromo c oxidasa.}} \\
	\hline
	\textbf{no.} & \textbf{Especie} & \textbf{COX 4} & \textbf{COX 5A} & \textbf{COX 5B} & \textbf{COX 6A2} \\
	\hline
	\endfirsthead
	\hline
	\textbf{no.} & \textbf{Especie} & \textbf{COX 4} & \textbf{COX 5A} & \textbf{COX 5B} & \textbf{COX 6A2} \\
	\hline
	\endhead
1 & \textit{Ailuropoda melanoleuca} & XM\_002913384.1 & XM\_002922971.1 & XM\_002912393.1 & XM\_002925805.1 \\ 
2 & \textit{Ateles belzebuth} & - & - & - & - \\ 
3 & \textit{Bos taurus} & NM\_001001439 & NM\_001002891.1 & NM\_001034046.2 & NM\_174522.2 \\ 
4 & \textit{Callicebus donacophilus} & - & DQ987248.1 & - & - \\ 
5 & \textit{Callithrix jacchus} & - & XM\_002753347.2 & - & XM\_002761936.2 \\ 
6 & \textit{Callithrix pygmaea} & - & DQ987246.1 & - & - \\ 
7 & \textit{Canis lupus familiaris} & XM\_536759.3 & XM\_535544.3 & NM\_001144130.1 & XM\_851113.2 \\ 
8 & \textit{Cavia porcellus} & XM\_003461023.1 & XM\_003463579.1 & XM\_003471692.1 & XM\_003477896.1 \\ 
9 & \textit{Cricetulus griseus} & XM\_003495107.1 & - & XM\_003498291.1 & XM\_003516063.1 \\ 
10 & \textit{Colobus guereza} & - & DQ987244.1 & - & - \\ 
11 & \textit{Equus caballus} & XM\_001502557.1 & XM\_001491856.3 & XM\_001493723.2 & XM\_001500332.1 \\ 
12 & \textit{Eulemur fulvus} & - & DQ987251.1 & - & - \\ 
13 & \textit{Gorilla gorilla} & - & DQ987240.1 & - & - \\ 
14 & \textit{Homo sapiens} & NM\_001861.3 & NM\_004255.3 & CR541727.1 & M83308.1 \\ 
15 & \textit{Loxodonta africana} & XM\_003418076.1 & XR\_133810.1 & XM\_003422005.1 & XM\_003418792.1 \\ 
16 & \textit{Macaca fascicularis} & - & - & - & - \\ 
17 & \textit{Macaca mulatta} & NM\_001193548.1 & NM\_001040279.1 & XM\_001098868.2 & NM\_001194578.1 \\ 
18 & \textit{Macaca silenus} & - & - & - & - \\ 
19 & \textit{Monodelphis domestica} & - & - & - & - \\ 
20 & \textit{Mus musculus} & BC132269.1 & NM\_007747.2 & NM\_009942.2 & U08439.1 \\ 
21 & \textit{Nomascus gabriellae} & - & DQ987242.1 & - & - \\ 
22 & \textit{Nomascus leucogenys} & XM\_003272506.1 & XM\_003267228.1 & XM\_003282429.1 & XM\_003280466.1 \\ 
23 & \textit{Nycticebus coucang} & - & DQ987249.1 & - & - \\ 
24 & \textit{Oryctolagus cuniculus} & NM\_001170882.1 & - & XM\_002710006.1 & XM\_002721729.1 \\ 
25 & \textit{Otolemur crassicaudatus} & - & DQ987250.1 & - & - \\ 
26 & \textit{Pan paniscus} & - & DQ987239.1 & - & XM\_001158801.2 \\ 
27 & \textit{Pan troglodytes} & NM\_001251915.1 & NM\_001118913.1 & XM\_001154903.2 & - \\ 
28 & \textit{Papio anubis} & - & DQ987245.1 & - & - \\ 
29 & \textit{Pongo abelii} & NM\_001133005.1 & XM\_002824314.1 & NM\_001131385.1 & NM\_001131893.1 \\ 
30 & \textit{Pongo pygmaeus} & - & DQ987241.1 & - & - \\ 
31 & \textit{Rattus norvegicus} & NM\_017202.1 & NM\_145783.1 & BC083179.1 & NM\_012812.3 \\ 
32 & \textit{Rousettus leschenaultii} & GU292805.1 & - & - & - \\ 
33 & \textit{Saguinus labiatus} & - & DQ987247.1 & - & - \\ 
34 & \textit{Saimiri sciureus} & - & AY585857.1 & - & - \\ 
35 & \textit{Sus scrofa} & AK399746.1 & XM\_003482240.1 & AK400056.1 & XM\_003481013.1 \\ 
36 & \textit{Symphalangus syndactylus} & - & DQ987243.1 & - & - \\ 
37 & \textit{Tarsius syrichta} & - & AY236506.1 & - & - \\ 
38 & \textit{Trachypithecus cristatus} & - & - & - & - \\ 
\bottomrule
\end{longtable}

\begin{longtable}{l l c c c}
	\caption{\label{TablanDNA2} \small{Continuaci�n de la tabla \ref{TablanDNA}.}} \\
	\hline
	\textbf{no.} & \textbf{Especie} & \textbf{COX 6B1} & \textbf{COX 6C} & \textbf{COX 7A1} \\
	\hline
	\endfirsthead
	\hline
	\textbf{no.} & \textbf{Especie} & \textbf{COX 6B1} & \textbf{COX 6C} & \textbf{COX 7A1} \\
	\hline
	\endhead
1 & \textit{Ailuropoda melanoleuca} & HQ380463.1 & XR\_097052.1 & XM\_002920932.1  \\ 
2 & \textit{Ateles belzebuth} & - & - & -  \\ 
3 & \textit{Bos taurus} & NM\_176675.3 & \textit{ XM\_002690379.1} &  BC114907.1  \\ 
4 & \textit{Callicebus donacophilus} & - & - & -  \\ 
5 & \textit{Callithrix jacchus} & XM\_002762021.2 & XM\_002759357.2 & XM\_002762060.2  \\
6 & \textit{Callithrix pygmaea} & - & - & -  \\ 
7 & \textit{Canis lupus familiaris} & XM\_003432596.1 & XM\_850997.2 & -  \\ 
8 & \textit{Cavia porcellus} & XM\_003467210.1 & XM\_003479873.1 & XM\_003467228.1  \\ 
9 & \textit{Cricetulus griseus} & XM\_003505555.1 & - & -  \\ 
10 & \textit{Colobus guereza} & - & - & -  \\ 
11 & \textit{Equus caballus} & XM\_001492385.3 & XM\_001492004.3 & XM\_001492779.2  \\ 
12 & \textit{Eulemur fulvus} & - & - & -  \\ 
13 & \textit{Gorilla gorilla} & - & - & -  \\ 
14 & \textit{Homo sapiens} & NG\_012193.1 & BC000187.2 & AK312091.1  \\ 
15 & \textit{Loxodonta africana} & XM\_003420756.1 & XM\_003408413.1 & XM\_003422300.1  \\ 
16 & \textit{Macaca fascicularis} & AB179393.1 & - & -  \\ 
17 & \textit{Macaca mulatta} & NM\_001040281.1 & XM\_001097441.2 & NM\_001040278.1  \\ 
18 & \textit{Macaca silenus} & - & AY236508.1 & -  \\ 
19 & \textit{Monodelphis domestica} & XM\_001363836.1 & - & -  \\ 
20 & \textit{Mus musculus} & NM\_025628.2 & NM\_053071.2 & BC060974.1  \\ 
21 & \textit{Nomascus gabriellae} & - & - & - \\ 
22 & \textit{Nomascus leucogenys} & XM\_003280044.1 & XM\_003262020.1 & -  \\ 
23 & \textit{Nycticebus coucang} & - & AY236512.1 & -   \\ 
24 & \textit{Oryctolagus cuniculus} & XM\_002722221.1 & XM\_002721825.1 & XM\_002722250.1  \\ 
25 & \textit{Otolemur crassicaudatus} & - & - & -  \\ 
26 & \textit{Pan paniscus} & - & - & -  \\ 
27 & \textit{Pan troglodytes} & XM\_001160605.1 & XM\_001151309.2 & XM\_003316334.1  \\ 
28 & \textit{Papio anubis} & - & - & -  \\ 
29 & \textit{Pongo abelii} & NM\_001131741.1 & XM\_002834759.1 & -  \\ 
30 & \textit{Pongo pygmaeus} & - & - & -  \\ 
31 & \textit{Rattus norvegicus} & NG\_028330.1 & BC058480.1 & XM\_001075627.2  \\ 
32 & \textit{Rousettus leschenaultii} & - & - & -  \\ 
33 & \textit{Saguinus labiatus} & - & - & -  \\ 
34 & \textit{Saimiri sciureus} & - & - & -  \\ 
35 & \textit{Sus scrofa} & NM\_001097497.1 & AK392221.1 & NM\_214411.1  \\ 
36 & \textit{Symphalangus syndactylus} & - & - & -  \\ 
37 & \textit{Tarsius syrichta} & AY236504.1 & AY236503.1 & AY585864.1  \\ 
38 & \textit{Trachypithecus cristatus} & - & AY236509.1 & -  \\ 
\bottomrule
\end{longtable}

\begin{longtable}{l l c c c }
	\caption{\label{TablanDNA3} \small{Continuaci�n de la tabla \ref{TablanDNA}.}} \\
	\hline
	\textbf{no.} & \textbf{Especie} & \textbf{COX 7B} & \textbf{COX 7C} & \textbf{COX 8C} \\
	\hline
	\endfirsthead
	\hline
	\textbf{no.} & \textbf{Especie} & \textbf{COX 7B} & \textbf{COX 7C} & \textbf{COX 8C} \\
	\hline
	\endhead
1 & \textit{Ailuropoda melanoleuca} & XM\_002930510.1 & XM\_002913621.1 & XM\_002930619.1  \\ 
2 & \textit{Ateles belzebuth} & - & - & AA910507.1  \\ 
3 & \textit{Bos taurus} & NM\_175795.3 & NM\_175831.3 & \textit{ NM\_001114517.2}  \\ 
4 & \textit{Callicebus donacophilus} & - & - & -  \\ 
5 & \textit{Callithrix jacchus} & XM\_003735770.1 & XM\_002744779.2 & XM\_002755707.1  \\ 
6 & \textit{Callithrix pygmaea} & - & - & -  \\ 
7 & \textit{Canis lupus familiaris} & XM\_003435588.1 & XM\_847363.2 & XM\_003639656.1  \\ 
8 & \textit{Cavia porcellus} & XM\_003470974.1 & XM\_003479466.1 & XM\_003461297.1  \\ 
9 & \textit{Cricetulus griseus} & - & - & XM\_003515383.1  \\ 
10 & \textit{Colobus guereza} & - & - & -  \\ 
11 & \textit{Equus caballus} & XM\_003365825.1 & XM\_003362849.1 & XM\_003362667.1  \\ 
12 & \textit{Eulemur fulvus} & - & - & AY254828.1  \\ 
13 & \textit{Gorilla gorilla} & - & - & -  \\ 
14 & \textit{Homo sapiens} & AK311879.1 & BC007498.2 & -  \\ 
15 & \textit{Loxodonta africana} & XM\_003412712.1 & - & -  \\ 
16 & \textit{Macaca fascicularis} & - & - & -  \\ 
17 & \textit{Macaca mulatta} & NM\_001258140.1 & XM\_001081984.2 & -  \\ 
18 & \textit{Macaca silenus} & - & - & -  \\ 
19 & \textit{Monodelphis domestica} & - & - & -  \\ 
20 & \textit{Mus musculus} & BC024350.1 & XM\_003689339.1 & BC086930.1  \\ 
21 & \textit{Nomascus gabriellae} & - & - & -  \\ 
22 & \textit{Nomascus leucogenys} & XM\_003282497.1 & XM\_003261573.1 & -  \\ 
23 & \textit{Nycticebus coucang} & - & - & -  \\ 
24 & \textit{Oryctolagus cuniculus} & XM\_002722477.1 & XM\_002721439.1 & XM\_002724105.1  \\ 
25 & \textit{Otolemur crassicaudatus} & - & - & -  \\ 
26 & \textit{Pan paniscus} & - & - & -  \\ 
27 & \textit{Pan troglodytes} & XR\_129485.1 & AK307009.1 & -  \\ 
28 & \textit{Papio anubis} & - & - & -  \\ 
29 & \textit{Pongo abelii} & XR\_092709.1 & XM\_002809590.1 & -  \\ 
30 & \textit{Pongo pygmaeus} & - & - & -  \\ 
31 & \textit{Rattus norvegicus} & FQ213156.1 & FQ224715.1 & FQ223790.1  \\ 
32 & \textit{Rousettus leschenaultii} & - & - & -  \\ 
33 & \textit{Saguinus labiatus} & - & - & -  \\ 
34 & \textit{Saimiri sciureus} & - & AY585860.1 & -  \\ 
35 & \textit{Sus scrofa} & AK392166.1 & DQ629155.1 & NM\_001097500.1  \\ 
36 & \textit{Symphalangus syndactylus} & - & - & -  \\ 
37 & \textit{Tarsius syrichta} & - & AY236505.1 & AY254827.1  \\ 
38 & \textit{Trachypithecus cristatus} & - & - & -  \\ 
\end{longtable}

\end{landscape}
% End Tabla Data 3 nDNA

\pagebreak
\newpage

\section{Informaci�n estructural}
Para los trabajos expuestos en los cap�tulos 2 y 3, se realizaron an�lisis sobre los modelos at�micos como 1be3 (complejo III) y 2occ (complejo IV) en \textit{Protein Data Bank} (PDB). En la tabla \ref{TablaPDBs} se resume alguna informaci�n de inter�s sobre sendos modelos, obtenida de la misma base de datos:

\begin{table}[!hbt]
\caption{\label{TablaPDBs}\small{Modelos at�micos analizados en esta tesis.}}
\begin{threeparttable}
	\begin{tabular}{c l c c c}
	\toprule
	Id. & Especie & T�cnica & Resoluci�n & Referencia \\
	\midrule
	1be3 & \textit{Bos taurus} & Dif. rayos X & 3.00\AA & \cite{Iwata:1998fj} \\
	& \multicolumn{4}{l}{Enlace: \url{http://bit.ly/1566Fb0}} \\
	2occ & \textit{Bos taurus} & Dif. rayos X & 2.30\AA & \cite{Yoshikawa:1998kx} \\
	& \multicolumn{4}{l}{Enlace: \url{http://bit.ly/1aBKbGU}} \\
	\bottomrule
	\end{tabular}
\end{threeparttable}
\end{table}

\section{Edad de las prote�nas del complejo IV}

En algunos an�lisis llevados a cabo en el cap�tulo 3, se computaron las edades (en millones de a�os) de las prote�nas codificadas por nDNA, presentes en el complejo citocromo c oxidasa. Estos datos fueron obtenidos mediante el servidor \emph{ProteinHistorian} \cite{Capra:2012bt}, cuya interfaz web est� disponible en: \url{http://lighthouse.ucsf.edu/ProteinHistorian/}. En la tabla \ref{TablaEdades} se muestra dicha informaci�n.

\begin{table}[!hbt]
\caption{\label{TablaEdades}\small{Edad de las cadenas del complejo IV codificadas por nDNA de \textit{Bos taurus}.}}
\begin{threeparttable}
	\begin{tabular}{l c c l c c}
	\toprule
	Subunidad & Edad (mill. a�os) & UniProtKB$^\S$ & Tax�n de origen \\
	\midrule
	COX 1 & 4200 & P00396 &  Bacteria\\
	COX 2 & 4200 & P68530 & Bacteria\\
	(*) COX 3 & 4200 & P00415 & Bacteria\\
	COX 4A & 1368 & P00423 & Opisthokonta \\
	COX 5A & 1368 & P00426 & Opisthokonta\\
	COX 5B & 1628 & P00428 & Eukaryota \\
	(\textdagger) COX 6A2 & ?? & P00471 & ?? \\
	COX 6B1 & 1368 & P00429 & Opisthokonta \\
	COX 6C & 842.0 & P04038 & Deuterostomia \\
	COX 7A1 & 176.1 & P07470 & Theria  \\
	COX 7B & 361.2 & P13183 & Tetrapoda \\
	COX 7C & 454.6 & P00430 & Euteleostomi \\
	(*) COX 8B & 910 & P10175 & Bilateria \\
	\bottomrule
	\end{tabular}
	\begin{tablenotes}
\item{\small{$\S$ Identificador de proteina en UniProt (\url{http://www.uniprot.org/}), el cual es el dato de entrada del servidor \emph{ProteinHistorian}.}}% Pie de tabla
\item{\small{\textdagger  La edad de la subunidad COX 6A2 fue imposible determinar por \emph{ProteinHistorian} por razones desconocidas.}}
\item{\small{* La edad de COX 3 y COX 8B fue estimada mediante un algoritmo distinto (\textit{Dollo parsimony}) al que se us� para estimar la edad del resto de las subunidades (\textit{Wagner parsimony}). V�ase \url{http://lighthouse.ucsf.edu/ProteinHistorian/methods.html} para m�s detalles sobre los m�todos de \textit{ProteinHistorian}.}}
\end{tablenotes}
\end{threeparttable}
\end{table}


% Activate the following line by filling in the right side. If for example the name of the root file is Main.tex, write
% "...root = Main.tex" if the chapter file is in the same directory, and "...root = ../Main.tex" if the chapter is in a subdirectory.

%!TEX root =  ../Tesis.tex

\chapter*{Agradecimientos}\addcontentsline{toc}{section}{Agradecimientos}
{
\raggedleft{
\textit{
<<
Dec�a Bernardo de Chartres que somos como enanos a los hombros de gigantes. Podemos ver m�s, y m�s lejos que ellos, no por alguna distinci�n f�sica nuestra, sino porque somos levantados por su gran altura.
>>
}
\\
\vspace{1cm}
Juan de Salisbury, \textit{Metalogicon} (1159).
}\\
\vspace{2cm}
}

El trabajo que se recoge en esta memoria  hubiera sido imposible sin la inestimable ayuda de mi director de tesis, Juan Carlos Aledo. Siempre presente y siempre dispuesto a resolver mis dudas en cualquier momento y lugar. Su firmeza, voluntad y perseverancia son, honestamente, admirables. Estas cualidades personales, sumadas a su incuestionable competencia como cient�fico, hacen de �l un gran maestro. Gracias, JC.

Mis otros \emph{gigantes} son, afortunadamente, mi familia. Aunque siempre han estado presentes a lo largo de mi vida acad�mica, en esta �ltima etapa he recibido m�s �nimos por su parte. Han sabido apoyarme justo cuando lo he necesitado y me han hecho saber que comparten conmigo la inminente satisfacci�n de defender con �xito este trabajo de tesis. Mi hermano David, quiz�s merece una menci�n especial por su implicaci�n pr�ctica en el trabajo de investigaci�n. Adem�s de poner a mi entera disposici�n sus servidores remotos y el asesoramiento que he necesitado (algo patente en el contenido de esta memoria), ha sabido resolverme un sinf�n de cuestiones de toda clase, matem�tica, inform�tica, etc., sin importarle cu�nto de su tiempo le he consumido. Muchas gracias, familia.

Tambi�n quiero agradecer a Jose �ngel Campos y Carolina Cardona, del laboratorio vecino, por ser esos grandes compa�eros y amigos del departamento, quienes, desde el primer momento, supieron depositar su confianza en m� sin esperar m�s a cambio. De igual modo lo hizo Carolina Lobo, quien ha sabido dar los mejores consejos en los mejores momentos. Ya sabes, Carito, que te debo un r�o con tus peques. Por su parte, Vero e Ian, recientes doctores, han marcado una senda con su ejemplo de seriedad, trabajo y disciplina. Gracias, pareja, por apoyarme en todo lo que ha estado en vuestra mano. Sinceramente, valoro mucho la atenci�n que he recibido por vuestra parte.

Cristina ha sido la persona que m�s tiempo ha pasado conmigo. Consecuentemente, ha hecho de sumidero de muchos momentos de estr�s. Su actitud y su predisposici�n a ayudar en todo lo que estuviera en su mano durante la etapa de escritura, ha sido determinante para llevar la tesis a buen fin. Del mismo modo, no puedo dejar de agradecer a sus padres, Carmencita y Manolo, por su desinteresada e inestimable ayuda en nuestro d�a a d�a. Gracias de todo coraz�n.

Aunque la suma de sus edades no llegue a los dedos de una sola mano, quisiera mencionar a Diego y a Sa�l. Quiz�s me di cuenta tarde, pero pasar tiempo con ellos era la mejor forma que encontr� de desconectar cuando deb�a hacerlo. Adem�s, a parte del sentido pr�ctico que, patol�gicamente intento darle a todo, bien saben los que me rodean que estos ni�os se han ganado todo mi cari�o. Gracias, peques.

Por �ltimo, y no menos importante, quisiera mencionar a todos aquellos amigos que han estado pendiente de mi trayectoria y han mostrado un inter�s sincero. Javi, Leo, Jessi, Vero, Sara, Lucas, Mara y Dani. Gracias, amigos.

% Activate the following line by filling in the right side. If for example the name of the root file is Main.tex, write
% "...root = Main.tex" if the chapter file is in the same directory, and "...root = ../Main.tex" if the chapter is in a subdirectory.

%!TEX root =  ../Tesis.tex
\newgeometry{top=16em}

\chapter*{Pr�logo}\addcontentsline{toc}{section}{Pr�logo}
El cuerpo de esta tesis doctoral est� compuesto por tres trabajos de investigaci�n, cuya motivaci�n com�n es la de describir las fuerzas evolutivas responsables de la variaci�n en la secuencia de prote�nas codificadas por el genoma mitocondrial de mam�feros. El primer trabajo, titulado \textit{``Mutational bias plays an important role in shaping longevity-related amino acid content in mammalian mtDNA-encoded proteins''}, aporta al estudio de la evoluci�n molecular un modelo computacional capaz de discernir entre los efectos de fuerzas evolutivas neutrales y selectivas sobre la abundancia de cada amino�cido en las prote�nas estudiadas. Adem�s, se centra en el an�lisis de la variaci�n de la cantidad de metionina en relaci�n con su potencial funci�n antioxidante. En el segundo, titulado \textit{``Thermodynamic stability explains the differential evolutionary dynamics of Cytochrome b and COX1 in mammals''}, se toman en consideraci�n dos propiedades estructurales: la exposici�n al solvente y la estabilidad termodin�mica; que influyen en el ritmo evolutivo de dos de las prote�nas codificadas por el genoma mitocondrial de mam�feros. Por �ltimo, el tercer trabajo de investigaci�n, titulado \textit{``Evolution of residues from the Cytochrome c Oxidase complex engaged in intermolecular contacts''}, se a�ade un nuevo elemento estructural que influye directamente en la din�mica evolutiva de las prote�nas: la relaci�n de proximidad entre regiones de la estructura de prote�nas que forman parte de un complejo enzim�tico, teniendo en cuenta, adem�s, el origen gen�mico de cada subunidad, poniendo as� de manifiesto fen�menos de coevoluci�n intergen�mica.


\restoregeometry